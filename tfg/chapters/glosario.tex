\chapter* {Anexo A. Glosario}

Acrónimos

\begin{tabbing}	
CCP \quad\quad\quad\quad\quad\= Colector Cilindro-Parabólico  \\
CSV \> Comma Separated Values \\
DNI \> Direct Normal Irradiance (Radiación Normal Directa, (($W/m^2$)) \\
HCE \> Heat Collector Element \\
HTF \> Heat Transfer Fluid \\
IAM \> Incidence Angle Modifier (Modificador por angulo de incidencia) \\
JSON \> JavaScript Object Notation \\
IPH \> Industrial Process Heat \\
NREL \> National Renewable Energy Laboratory \\
OOP \> Object Oriented Programming \\
POO  \> Programación Orientada a Objetos \\
PTC \> Parabolic Trough Collector \\
SAM\> System Advisor Model \\
SCA \> Solar Collector Assembly\\
SCE \> Solar Collector Element \\
SPA \> Solar Position Algorithm \\
TFG	\>Trabajo de Fin de Grado \\
TMY \> Typical Meteorological Year \\
UNED \> Universidad Nacional de Educación a Distancia \\
\end{tabbing}


Símbolos latinos

\begin{tabbing}
$A_0, A_1$ \quad\quad\quad\quad\quad\= constantes del modelo de emisividad equivalente para receptores, (-) \\
$A_{c}$ \> superficie de apertura del concentrador, ($m^2$) \\
$A_{cs,b}$ \> sección transversal de la unión entre el soporte y el tubo absorbedor, ($m^2$) \\
$A_{ext}$ \>  área exterior del receptor expuesta a la radiación solar, $(m^2)$ \\
$A_{p}$ \> Apertura de la superficie parabólica, (m) \\
$Cg$  \>   factor de concentración \\
$c_P$ \> calor específico a presión constante, $(J/kg\cdot K)$ \\
$D_{gi}$  \>  diámetro interior de la envolvente de vidrio del tubo absorbedor (m) \\
$D_{go}$ \> diámetro exterior de la envolvente de vidrio del tubo absorbedor (m)   \\
$D_L$ \> distancia de separación entre lazos, (m) \\
$D_{ri}$  \>  diámetro interior del tubo absorbedor (m) \\
$D_{ro}$ \> diámetro exterior del tubo absorbedor (m)   \\
$F_0, F_1$ \> constantes para el cálculo del IAM según la ec.(\eqref{eq:iam}), (-) \\
$F'_{crit}$ \> coeficiente adimensional crítico del modelo segúnla ec.(\eqref{eq:fcrit}) \\
$f_0, f_1, f_2, f_3, f_4$ \> factores adimensionales del modelo según las ecuaciones \eqref{eq:f0}, \eqref{eq:f1}-\eqref{eq:f4} \\
$f_l$ \> distancia focal, (m) \\
$g(Z)$ \> función característica del rendimiento térmicos según la ec.\eqref{eq:gdezeta}, (-) \\
$g', g'', g''', g^{IV}$ \> derivadas de orden 1 a 4 de la función característica $g(Z)$, (-) \\
$h(T)$ \> entalpía específica, $(J/kg)$ \\
$\dot H$ \> tasa temporal de variación de la entalpía del HTF en el sistema, $(J/s)$ \\
$\bar h_b$ \>coeficiente de transmisión por convección medio en el soporte, ($W/(m^2 \cdot K)$) \\
$h_{ext}$ \> coeficiente de transferencia de calor convectivo equivalente \\
$h_{int}$ \> coeficiente de transferencia de calor convectivo hacia el interior, $(W/(m^{2}\cdot K))$ \\
$k_b$ \> conductividad térmica del acero del brazo soporte del receptor,  $(W/(m\cdot K))$ \\
$k_f$ \> conductividad térmica a la temperatura del fluido, $(W/(m\cdot K))$ \\
$k_{rec}$ \> conductividad del material del receptor, $(W/(m\cdot K))$ \\
$k_T$ \> conductividad térmica, $(W/m \cdot K)$ \\
$L$  \> longitud del tubo absorbedor (m) \\
$L_{bordes}$  \> longitud del tubo absorbedor que no recibe radiación, (m) \\
$\dot m$ \> caudal másico del HTF, (kg/s) \\
$NTU_{perd}$ \> número de Unidades de Transmisión, (-) \\
$NTU_{perd}$ \> número de Unidades de Transmisión definido según la ec.\eqref{eq:ntuperd}, (-) \\
$Nu_{G}$ \> número de Nusselt calculado con la correlación de Gnielinski, ec.($\eqref{eq:nug}$), (-) \\
$P_b$ \> perímetro del soporte del receptor, (m) \\
$Pr_f$ \> número de Prandtl a la temperatura del fluido, (-) \\
$Pr_{ri}$ \> número de Prandtl a la temperatura de la pared interior del receptor, (-) \\
$\dot q_{abs}$ \> potencia calorífica absorbida a lo largo de todo el HCE, (W) \\
$\dot q''_{abs}$ \> flujo de radiación absorbido por la superficie exterior del receptor, $(W/m^{2})$ \\
$\dot q_{perd}$ \> potencia calorífica perdida a lo largo de todo el HCE, (W) \\
$\dot q_{perd,soportes}$ \> potencia calorífica perdida a través de los soportes del HCE, (W) \\
$\dot q''_{perd}(x)$ \> flujo de calor de pérdidas al exterior desde la superficie del receptor, $(W/m^{2})$ \\
$\dot q"_{u}(x)$ \> flujo de calor útil hacia el interior del receptor,  $(W/m^{2})$ \\ 
$T_{base}$ \> temperatura de la zona de conexión entre los brazos y el tubo absorbedor, (K) \\
$T_{ext}$ \> temperatura ambiente exterior, (K) \\
$T_{f}$ \> temperatura del fluido, (K) \\
$\bar{T}_{f}$ \> temperatura media del fluido, (K) \\
$T(h)$ \> temperatura en función de la entalpía, (K) \\
$T_{in}$ \> temperatura del fluido a la entrada del receptor, (K) \\
$T_{in}$ \>  temperatura del fluido a la entrada del receptor, (K) \\
$T_{out}$ \> temperatura del fluido a la salida del receptor, (K) \\
$T_{ro}(x)$ \> temperatura de pared exterior, (K) \\
$U_{crit}$ \>  coeficiente de transmisión de calor al interior crítico según la ec.\eqref{eq:ucrit}, $(W/m^{2}\cdot K)$  \\
$U_{ext}$ \> coeficiente de transmisión de calor al exterior $(W/m^{2}\cdot K)$  \\
$U_{rec}$  \> coeficiente global de transferencia de calor hacia el interior $(W/m^{2}\cdot K)$  \\
$W_{spd}$ \> velocidad del viento, (m/s) \\
$x$ \> coordenada longitudinal, (m) \\
$x^*$ \> coordenada longitudinal adimiensional en función de la longitud del tubo, (-) \\
$Z$ \> variable adimiensional del modelo de rendimiento térmico según la ec.\eqref{eq:zeta}, (-) \\
\end{tabbing}


Símbolos griegos

\begin{tabbing}
$\alpha$ \quad\quad\quad\quad\quad\= absortividad del receptor, (-) \\
$\beta$ \> ángulo de seguimiento, (rad) \\
$\gamma$ \> fracción solar, (-) \\
$\gamma_L$ \>  factor de longitud efectiva, (-)  \\
$\gamma_g$  \> factor de interceptación geométrico, (-)  \\
$\varepsilon_{ext}$ \> emisividad equivalente de la superficie exterior del tubo, (-) \\
$\eta_{bordes}$ \> coeficiente de pérdidas geométricas, (-)  \\
$\eta_{disponibilidad}$ \> factor de disponibilidad, (-) \\
$\eta_{geométrico}$ \> factor geométrico, (-) \\
$\eta_{opt}(\theta)$  \> rendimiento óptico, (-)  \\
$\eta_{opt,pico}(\theta)$  \> rendimiento óptico pico, (-)  \\
$\eta_{seguidor}$ \> factor de precisión del seguidor, (-) \\
$\eta_{sombras}$ \> coeficiente de pérdidas por sombreado, (-)  \\
$\eta_{suciedad}$ \> factor de suciedad, (-) \\
$\eta_{T}$ \> rendimiento para la totalidad del receptor en el modelo simplificado, (-)  \\
$\eta(x)$ \> rendimiento integral hasta una distancia $x$ de la entrada, (-)  \\
$\eta_x(x)$ \> rendimiento local en una sección a distancia $x$ de la entrada, (-) \\
$\theta$ \> ángulo de incidencia, (rad) \\
$\mu(T)$ \> viscosidad dinámica, $(Pa \cdot m)$ \\
$\rho$ \> reflectividad, (-) \\
$\rho(T)$ \> densidad $(kg/m^3)$ \\
$\sigma$ \> constante de Stefan-Boltzmann $(5,67x10^{-8} W/(m^2 \cdot K^4))$\\
$\tau$ \> transmisividad de la cubierta de vidrio del receptor, (-) 
\end{tabbing}

