\chapter{Análisis del modelo aplicado a un central solar termoeléctrica real}
\label{analisis-central}

En este capítulo realizaremos la simulación a partir de datos reales (meteorológicos y de generación) de una central termosolar. Puesto que a lo largo de todo un año de generación se producen una serie de eventos que desvían el comportamiento de la planta del considerado en las fases de diseño, esta comparación nos servirá para determinar mejor el alcance y posibilidades que el código de simulación puede tener. Trataremos de identificar cómo condiciona el acoplamento entre el campo sola y el ciclo de potencia la operación del primero con vistas a desarrollos futuros de nuevos módulos de simulación que nos permitan ampliar nuestro código. 

\section{Descripción de la simulación de la central solar termoeléctrica}
\label{descripcion-central}

La configuración de campo solar que se va a utilizar a lo largo de las siguientes simulaciones está basada en las plantas termosolares Aste 1A y 1B, que se encuentran situadas en el término municipal de Alcázar de San Juan, provincia de Ciudad Real. Sus coordenadas geográficas son (39,1°N 3,16°W) y la altitud es de 651 m sobre el nivel del mar.

[FOTO AEREA ASTE]

La potencia eléctrica nominal de cada una de ellas es de 49,9 MW. El proyecto inicial consideraba que las plantas contarían con almacenamiento térmico, el cual se construiría durante una segunda fase que finalmente no se llegó a ejecutar, por lo que en la actualidad solo existe generación durante las horas de sol. Se emplearán los datos de Aste 1B, cuya configuración es la siguente:

El campo solar cuenta con 120 lazos distribuidos de manera irregular en 4 subcampos. Consideraremos que existe un único subcampo. La  distancia de separación entre lazos es de 16,25 m. 

\begin{itemize}[itemsep=2pt,parsep=2pt]
\item
  Subcampo NO, 31 lazos.
\item
  Subcampo NE, 28 lazos.
\item
  Subcampo SO, 27 lazos.
\item
  Subcampo SE, 34 lazos.
\end{itemize}

Todos los lazos son idénticos, contando con 4 SCAs cada uno en una configuración tipo \(U\). El eje de seguimiento perfectamente plano se encuentra alineado en direccion N-S. Cada SCA cuenta con un total de 336 espejos de vídrio fabricados por Flabeg.


La configuración del campo solar ya se ha descrito en el apartado \ref{configuracion-del-campo}, pero en este caso el fluido de trabajo es Dowtherm A, cuyas propiedades también se han descrito en \ref{subclases-fluid}.

Los datos meteorológicos son los recogidos a lo largo durante 2016 por las tres estaciones meteorológicas con las que cuenta la planta. Al tener por triplicado las medidas de cada variable se adopta el criterio de seleccionar la mediana de las tres y no el valor medio. Esta selección la realiza el sistema de control de planta en cada momento y con este criterio se persigue conseguir una mayor robustez del sistema, pues si una estación presenta valores muy desviados de las otras dos podria darse el caso de que el valor medio estuviese muy alejado del valor medio real. Cuando por avería o fallo de comunicación se carece de los datos de alguna estación sí se suele adoptar el criterio de seleccionar el valor medio de las dos restantes.

El año 2016 es bisiesto y por tanto hemos suprimido el día 29 de febrero con el fin de poder introducir los datos en SAM, que no contempla el uso de datos de este tipo de años.

Para realizar nuestra simulación utilizaremos también los datos de caudal y temperatura de entrada en cada una de los subcampos registrados por los instrumentos de planta a la entrada de cada subcampo. Los datos de temperatura a la salida de los subcampos nos permitirán calcular la potencia térmica real generada por el campo solar que compararemos con la calculada por nuestro código y por SAM. 

\subsection{Resultados de la simulación}



 

