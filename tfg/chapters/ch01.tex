\chapter{Introducción}
\label{introduccion}

\section{Objetivo}
El propósito principal de este Trabajo Final de Grado (TFG) es profundizar en el conocimiento del funcionamiento de un campo solar de concentradores cilindroparabólicos (CCP) mediante el desarrollo de una herramienta de simulación programada en Python 3. Para ello, se ha partido del modelo teórico desarrollado en su Tesis Doctoral por el profesor de la Universidad Nacional de Educación a Distancia (UNED), el Dr.~Rubén Barbero Fresno y se ha empleado una metodología basada en el paradigma de la programación orientada a objetos (POO). El desarrollo de este trabajo ha supuesto un reto personal por la necesidad de adquirir habilidades en el manejo de diferentes herramientas antes desconocidas para mí, como el lenguaje de programación Python 3 \cite{WelcomePythonOrg} y sus diferentes librerías para el cálculo científico, concretamente para la resolución de ecuaciones mediante métodos numéricos (\emph{Numpy}, \cite{NumPy}) y el tratamiento masivo de datos (\emph{Pandas},  \cite{PandasPythonData}). Finalmente, todo el código se encuentra publicado y accesible a través de \emph{GitHub} y se pretende crear una versión interactiva online mediante \emph{Notebook Jupyter} (en proceso).

El propósito inicial era crear un código con un limitado número de Clases para el modelado del HCE pero, según el trabajo iba avanzando, se amplió para cubrir la necesidad de modelar también los SCA, el lazo completo, subcampos y campo solar completo. También el fluido caloportador ha merecido especial atención con el fin de que sea posible aprovechar la existencia de librerías software como \emph{CoolProp} \cite{bellPurePseudopureFluid2014} para la obtención  de las propiedades de los fluidos. Los  mecanismos de entrada y salida de datos son lo suficientemente flexibles como para permitir la lectura de archivos meteorológicos en formato \emph{TMY} y \emph{CSV}, para lo que se aprovecha parte las librerías desarrolladas para el código de simulación de sistemas fotovoltaicos, \emph{pvlib-python}, \cite{holmgrenPvlibPythonPython2018}.

Al mismo tiempo, se vio la necesidad de que las configuraciones de estos elementos permaneciesen guardas en ficheros que facilitasen su recuperación posteriormente, a modo de base de datos, pero que también fuesen fácilmente editables, con el fin de que el usuarios pudiera configurar sus propios modelos con un simple editor de texto. Por este motivo se optó por el formato \emph{JSON} para guardar la configuración de las simulaciones y de los sistemas que conforman el campo solar.

Para que el usuario no tenga que partir de cero a la hora de crear una configuración, se ha desarrollado también una sencilla interfaz gráfica con \emph{TkInter} \cite{TkInterPythonWiki}  que ayuda en el proceso de creación del archivo de configuración \emph{JSON}. Esta interfaz permite cargar una configuración ya creada o comenzar a partir de una plantilla. El usuario puedo guardar la nueva configuración con el nombre que desee para usarla posteriormente.

Los resultados obtenidos han sido verificados comparándolos con los obtenidos mediante System Advisor Model (SAM, \cite{freemanSystemAdvisorModel2018}). Los cálculos relativos a la posición solar y óptica geométrica presentan valores casi idénticos, con diferencias despreciables achacables a los redondeos propios del cálculo digital. Los cálculos relativos a rendimientos térmicos y otra variables termodinámicas presentan algunas diferencias, como es de esperar por el hecho de emplear modelos diferentes. Pero en todo caso, los resultados obtenidos son compatibles con los ofrecidos por SAM, teniendo en cuenta que en este proyecto solo se aborda la simulación del sistema del campo solar, el único sistema dentro del alcance del modelo teórico de partida. 

Finalmente, también se ha realizado un comparativa con los datos de generación de una planta termosolar real. En este caso, debido a las peculiaridades de funcionamiento de la planta real, cuyo campo solar está muy sobredimensionado, solo podríamos comparar la potencia térmica de salida del campo en aquellos momentos en los que supiésemos que no se están aplicando limitaciones de generación al campo. No obstante, sí hemos podido emplear nuestro código para estimar la magnitud de este sobredimensionado y exceso de energía térmica disponible en el campo. 

La metodología seguida permite que en el futuro este proyecto pueda ser ampliado de forma sistemática con la incorporación de nuevas clases de objetos que aprovechen los métodos de entrada y salida ya programados para interactuar con ellos. En todo caso, se han empleado valores de rendimientos estimados para los principales subsistemas de planta con el fin de ofrecer una estimación de la energía eléctrica finalmente vertida a la red.


\section{Estructura de la memoria}
Esta memoria se estructura en 5 capítulos y un anexo con el código fuente.

En este primer capítulo se describen los objetivos del TFG y se ofrece una introducción a las características de los concentradores cilindroparabólicos. 
El segundo capítulo presenta el modelo teórico de partida y se detallan las ecuaciones de los modelos para el cálculo del rendimiento térmico de los tubos absorbedores que más adelante serán modelados.
El tercer capítulo aborda el modelado de los diferentes sistemas necesarios para la caracterización del campo solar y se procede a la verificación por comparación con otra herramienta de simulación.
En el cuarto capítulo se desarrollan, a modo de ejemplo de aplicación, algunos análisis paramétricos de diferente tipo. También se contrastan los resultados de la simulación del campo solar con los datos disponibles de una planta termosolar real.
Finalmente, en el quinto capítulo se presentan algunas conclusiones y propuestas de desarrollo futuro.

\section{Concentradores cilindroparabólicos}

Existen principalmente cuatro tecnologías para el aprovechamiento de la radiación solar directa en sistemas térmicos: concentrador cilindro-parabólico (CCP), central de torre, concentrador Fresnel y disco parabólico. La tecnología de CCP es la que cuenta en la actualidad con una mayor madurez y gran número de centrales en operación y en construcción en todo el mundo \cite{islamComprehensiveReviewStateoftheart2018}. La tecnología de torre central ha crecido en los últimos años, pero todavía está lejos de la madurez de la tecnología de CCP. En España, actualmente hay medio centenar de centrales CCP en operación \cite{Protermosolar}, alguna de ellas desde hace más de una década, quedando probado que el estado del arte y la madurez de la tecnología garantizan el correcto funcionamiento del sistema. Este tipo de sistema presenta un alto grado de replicabilidad, modularidad y aprovechamiento del terreno. Desde el punto de vista económico, esta tecnología también resulta muy favorable ya que los costes de inversión y operación han sido comercialmente probados, al menos, para los sistemas termoeléctricos.

\begin{figure}[H]
 \centering
  \subfloat[Central Solar de Torre. Fuente \cite{Protermosolar}]{
   \label{f:torre}
    \includegraphics[width=0.48\textwidth, height=0.4\textwidth]{images/torre.png}}
  \subfloat[Concentradores de Disco Parabólico. Fuente \cite{Protermosolar}]{
   \label{f:disco}
    \includegraphics[width=0.48\textwidth, height=0.4\textwidth]{images/disco.png}}
  \\
   \subfloat[Colector Cilindro-Parabólico. Fuente: CST Aste 1A]{
   \label{f:ccp}
    \includegraphics[width=0.48\textwidth, height=0.4\textwidth]{images/ccp.png}}
   \subfloat[Concentrador de tipo Fresnel. Fuente \cite{Protermosolar}]{
   \label{f:fresnel}
    \includegraphics[width=0.48\textwidth, height=0.4\textwidth]{images/fresnel.png}}
 \caption{Tecnologías solares de concentración}
 \label{f:tecnologias_concentracion}
\end{figure}


En un CCP pueden distinguirse cuatro elementos principales: el reflector o concentrador, el tubo absorbedor, el sistema de seguimiento y el fluido caloportador.

\subsection{El concentrador cilindroparabólico}
\label{concentrador}
Un CCP consiste en una superficie a modo canal  de sección parabólica que refleja la radiación solar directa concentrándola sobre un tubo absorbedor colocado en la línea focal del paraboloide. Dentro de los diferentes sistemas de concentración solar pertenece al grupo de los concentradores lineales, al igual que los sistemas de concentración tipo Fresnel y al contrario que los sistemas de concentración de torre central o de discos parabólicos, en cuyo caso estaríamos hablando de sistemas de concentración puntuales. Por el tubo absorbedor se puede hacer circular algún fluido que se calentará debido a la radiación incidente sobre el tubo.  Se trata de una transformación directa de radiación solar en energía térmica con una buena eficiencia y que puede alcanzar temperaturas de hasta 675 K con los aceites sintéticos actuales. Se podrían alcanzar temperaturas mayores con sales fundidas o gases, pero el empleo de estos fluidos caloportadores todavía no cuenta con experiencias comerciales a gran escala.

\begin{figure}[H]
\includegraphics[width=0.9\linewidth]{images/entrada_salida_lazo2_texto.png}
\caption[Lazo de concentradores cilindroparabóicos]{Perspectiva de un lazo completo. Se indica el sentido del recorrido del HTF en el lazo, desde la tubería de entrada hasta la de salida. Fuente: CST Aste 1A} 
\label{fig:entrada_salida_lazo}
\end{figure}


\begin{figure}[H]
\includegraphics[width=0.7\linewidth, , height=0.9\textwidth]{images/perfil_sca_2_texto.png}
\caption[Vista de perfil de un SCA]{Vista de perfil de un SCA. Pueden distinguirse algunos HCEs y los brazos de soporte. Fuente: CST Aste 1A} 
\label{fig:perfil_sca}
\end{figure}

\subsection{El tubo absorbedor o receptor}
\label{tuboabsorbedor}
A lo largo del eje focal del concentrador se instala una conducción por la que circula un fluido caloportador o transmisor del calor (HTF por sus siglas en ingles, Heat Transfer Fluid). Esta conducción está compuesta en realidad por una serie de elementos tubulares denominados Heat Collector Element, HCE. Los HCE consisten en tubo de acero con una envolvente de vidrio de tal forma que en el proceso de fabricación se ha dejado extraído el aire que queda entre ambos (región anular o \textit{annulus}). De esta forma se reducen las pérdidas de calor por convección a través de la región anular. La soldadura vidrio-metal y unos elementos denominados \textit{getters} que absorben, hasta cierto punto, algunas moléculas que puedan filtrarse a la región anular durante la vida de operación del HCE permiten que éste cuente con pérdidas reducidas de calor mientras no se produzca la rotura del vidrio o la saturación de dichos \textit{getters}. 

\subsection{El sistema de seguimiento}
\label{sistemadeseguimiento}
Para que se produzca la concentración de la radiación solar incidente ésta debe ser perpendicular al eje que pasa por el foco y la base de la parábola. La primera consecuencia es que solo puede aprovecharse plenamente la componente normal de la radiación solar incidente (DNI). Dado que el Sol varía su posición relativa al concentrador continuamente, el conjunto reflector-tubo absorbedor está montado sobre una estructura que pueda girar sobre un eje con el fin de segir la trayectoria solar a lo largo del día. Salvo instalaciones especiales en laboratorios, no se emplean seguidores a dos ejes pues la complejidad de los colectores con movimiento basado en dos ejes es tal que no permite su rentabilidad debido a los costes de mantenimiento. Por tanto, el sistema de seguimiento más empleado consiste en mover la estructura del colector con un grado de libertad en torno a un eje que, en las plantas solares cuyo objetivo es maximizar el vertido anual de energía eléctrica a la red, cuenta con una orientación Norte-Sur, lo cual lleva a que exista una importante diferencia entre la generación en los meses de verano y los meses de invierno, siendo mayor en los primeros. Si lo que se persigue es obtener una producción más estable a lo largo del año, la orientación más adecuada del eje sería Este-Oeste.

La rotación del colector requiere un mecanismo de accionamiento, eléctrico o hidráulico,
dependiendo en muchos casos de las dimensiones y el peso de los elementos del colector. Para abaratar costes se suele emplear un mismo mecanismo para mover varios módulos.

El control del movimiento se puede llevar a cabo de forma autónoma, en el propio colector, dotándolo de algún dispositivo para detectar la posición del Sol en el cielo. Otra opción es emplear algoritmos matemáticos que calculan la posición del Sol para cada momento del día, en cualquier día del año. Una vez calculada la posición solar, se mueve el colector hasta colocarlo correctamente orientado. Este método requiere algún sistema para conocer la posición exacta del colector. Lo normal es emplear un codificador angular.
Dado que el colector solar se encuentra en movimiento, las conexiones del tubo absorbedor con las tuberías de entrada y salida de éste deben permitir el giro en los puntos de unión. Para esto se emplean conexiones flexibles y juntas rotativas combinadas adecuadamente. El coste de estos elementos es también elevado por lo que la elección de una configuración adecuada puede suponer un importante ahorro de costes en la instalación y el mantenimiento.

\subsection{El fluido caloportador}
\label{fluidocaloportador}

Actualmente, el rango de temperatura de trabajo con colectores cilindro parabólicos es de 425 K a 675 K.  El agua desmineralizada es una buena opción para temperaturas inferiores a los 450 K. A mayor temperatura es preferible el aceite sintético debido a que no aumenta tanto su presión. Si el campo solar está acoplado a un ciclo termodinámico para generación de energía eléctrica, se podría maximizar el rendimiento alcanzando temperaturas más altas, pero no hay ningún fluido que pueda dar unas prestaciones tecnológicas adecuadas a temperaturas superiores sin un aumento significativo del coste. Actualmente se están desarrollando algunas experiencias con sales fundidas y están en desarrollo sistemas de generación directa de vapor (emplean directamente el agua como fluido caloportador). Para temperaturas inferiores hay otros colectores más económicos (colectores de placa plana y colectores de tubo de vacío). 
