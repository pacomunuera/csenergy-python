\documentclass[11pt]{article}

    \usepackage[breakable]{tcolorbox}
    \usepackage{parskip} % Stop auto-indenting (to mimic markdown behaviour)
    
    \usepackage{iftex}
    \ifPDFTeX
    	\usepackage[T1]{fontenc}
    	\usepackage{mathpazo}
    \else
    	\usepackage{fontspec}
    \fi

    % Basic figure setup, for now with no caption control since it's done
    % automatically by Pandoc (which extracts ![](path) syntax from Markdown).
    \usepackage{graphicx}
    % Maintain compatibility with old templates. Remove in nbconvert 6.0
    \let\Oldincludegraphics\includegraphics
    % Ensure that by default, figures have no caption (until we provide a
    % proper Figure object with a Caption API and a way to capture that
    % in the conversion process - todo).
    \usepackage{caption}
    \DeclareCaptionFormat{nocaption}{}
    \captionsetup{format=nocaption,aboveskip=0pt,belowskip=0pt}

    \usepackage[Export]{adjustbox} % Used to constrain images to a maximum size
    \adjustboxset{max size={0.9\linewidth}{0.9\paperheight}}
    \usepackage{float}
    \floatplacement{figure}{H} % forces figures to be placed at the correct location
    \usepackage{xcolor} % Allow colors to be defined
    \usepackage{enumerate} % Needed for markdown enumerations to work
    \usepackage{geometry} % Used to adjust the document margins
    \usepackage{amsmath} % Equations
    \usepackage{amssymb} % Equations
    \usepackage{textcomp} % defines textquotesingle
    % Hack from http://tex.stackexchange.com/a/47451/13684:
    \AtBeginDocument{%
        \def\PYZsq{\textquotesingle}% Upright quotes in Pygmentized code
    }
    \usepackage{upquote} % Upright quotes for verbatim code
    \usepackage{eurosym} % defines \euro
    \usepackage[mathletters]{ucs} % Extended unicode (utf-8) support
    \usepackage{fancyvrb} % verbatim replacement that allows latex
    \usepackage{grffile} % extends the file name processing of package graphics 
                         % to support a larger range
    \makeatletter % fix for grffile with XeLaTeX
    \def\Gread@@xetex#1{%
      \IfFileExists{"\Gin@base".bb}%
      {\Gread@eps{\Gin@base.bb}}%
      {\Gread@@xetex@aux#1}%
    }
    \makeatother

    % The hyperref package gives us a pdf with properly built
    % internal navigation ('pdf bookmarks' for the table of contents,
    % internal cross-reference links, web links for URLs, etc.)
    \usepackage{hyperref}
    % The default LaTeX title has an obnoxious amount of whitespace. By default,
    % titling removes some of it. It also provides customization options.
    \usepackage{titling}
    \usepackage{longtable} % longtable support required by pandoc >1.10
    \usepackage{booktabs}  % table support for pandoc > 1.12.2
    \usepackage[inline]{enumitem} % IRkernel/repr support (it uses the enumerate* environment)
    \usepackage[normalem]{ulem} % ulem is needed to support strikethroughs (\sout)
                                % normalem makes italics be italics, not underlines
    \usepackage{mathrsfs}
    

    
    % Colors for the hyperref package
    \definecolor{urlcolor}{rgb}{0,.145,.698}
    \definecolor{linkcolor}{rgb}{.71,0.21,0.01}
    \definecolor{citecolor}{rgb}{.12,.54,.11}

    % ANSI colors
    \definecolor{ansi-black}{HTML}{3E424D}
    \definecolor{ansi-black-intense}{HTML}{282C36}
    \definecolor{ansi-red}{HTML}{E75C58}
    \definecolor{ansi-red-intense}{HTML}{B22B31}
    \definecolor{ansi-green}{HTML}{00A250}
    \definecolor{ansi-green-intense}{HTML}{007427}
    \definecolor{ansi-yellow}{HTML}{DDB62B}
    \definecolor{ansi-yellow-intense}{HTML}{B27D12}
    \definecolor{ansi-blue}{HTML}{208FFB}
    \definecolor{ansi-blue-intense}{HTML}{0065CA}
    \definecolor{ansi-magenta}{HTML}{D160C4}
    \definecolor{ansi-magenta-intense}{HTML}{A03196}
    \definecolor{ansi-cyan}{HTML}{60C6C8}
    \definecolor{ansi-cyan-intense}{HTML}{258F8F}
    \definecolor{ansi-white}{HTML}{C5C1B4}
    \definecolor{ansi-white-intense}{HTML}{A1A6B2}
    \definecolor{ansi-default-inverse-fg}{HTML}{FFFFFF}
    \definecolor{ansi-default-inverse-bg}{HTML}{000000}

    % commands and environments needed by pandoc snippets
    % extracted from the output of `pandoc -s`
    \providecommand{\tightlist}{%
      \setlength{\itemsep}{0pt}\setlength{\parskip}{0pt}}
    \DefineVerbatimEnvironment{Highlighting}{Verbatim}{commandchars=\\\{\}}
    % Add ',fontsize=\small' for more characters per line
    \newenvironment{Shaded}{}{}
    \newcommand{\KeywordTok}[1]{\textcolor[rgb]{0.00,0.44,0.13}{\textbf{{#1}}}}
    \newcommand{\DataTypeTok}[1]{\textcolor[rgb]{0.56,0.13,0.00}{{#1}}}
    \newcommand{\DecValTok}[1]{\textcolor[rgb]{0.25,0.63,0.44}{{#1}}}
    \newcommand{\BaseNTok}[1]{\textcolor[rgb]{0.25,0.63,0.44}{{#1}}}
    \newcommand{\FloatTok}[1]{\textcolor[rgb]{0.25,0.63,0.44}{{#1}}}
    \newcommand{\CharTok}[1]{\textcolor[rgb]{0.25,0.44,0.63}{{#1}}}
    \newcommand{\StringTok}[1]{\textcolor[rgb]{0.25,0.44,0.63}{{#1}}}
    \newcommand{\CommentTok}[1]{\textcolor[rgb]{0.38,0.63,0.69}{\textit{{#1}}}}
    \newcommand{\OtherTok}[1]{\textcolor[rgb]{0.00,0.44,0.13}{{#1}}}
    \newcommand{\AlertTok}[1]{\textcolor[rgb]{1.00,0.00,0.00}{\textbf{{#1}}}}
    \newcommand{\FunctionTok}[1]{\textcolor[rgb]{0.02,0.16,0.49}{{#1}}}
    \newcommand{\RegionMarkerTok}[1]{{#1}}
    \newcommand{\ErrorTok}[1]{\textcolor[rgb]{1.00,0.00,0.00}{\textbf{{#1}}}}
    \newcommand{\NormalTok}[1]{{#1}}
    
    % Additional commands for more recent versions of Pandoc
    \newcommand{\ConstantTok}[1]{\textcolor[rgb]{0.53,0.00,0.00}{{#1}}}
    \newcommand{\SpecialCharTok}[1]{\textcolor[rgb]{0.25,0.44,0.63}{{#1}}}
    \newcommand{\VerbatimStringTok}[1]{\textcolor[rgb]{0.25,0.44,0.63}{{#1}}}
    \newcommand{\SpecialStringTok}[1]{\textcolor[rgb]{0.73,0.40,0.53}{{#1}}}
    \newcommand{\ImportTok}[1]{{#1}}
    \newcommand{\DocumentationTok}[1]{\textcolor[rgb]{0.73,0.13,0.13}{\textit{{#1}}}}
    \newcommand{\AnnotationTok}[1]{\textcolor[rgb]{0.38,0.63,0.69}{\textbf{\textit{{#1}}}}}
    \newcommand{\CommentVarTok}[1]{\textcolor[rgb]{0.38,0.63,0.69}{\textbf{\textit{{#1}}}}}
    \newcommand{\VariableTok}[1]{\textcolor[rgb]{0.10,0.09,0.49}{{#1}}}
    \newcommand{\ControlFlowTok}[1]{\textcolor[rgb]{0.00,0.44,0.13}{\textbf{{#1}}}}
    \newcommand{\OperatorTok}[1]{\textcolor[rgb]{0.40,0.40,0.40}{{#1}}}
    \newcommand{\BuiltInTok}[1]{{#1}}
    \newcommand{\ExtensionTok}[1]{{#1}}
    \newcommand{\PreprocessorTok}[1]{\textcolor[rgb]{0.74,0.48,0.00}{{#1}}}
    \newcommand{\AttributeTok}[1]{\textcolor[rgb]{0.49,0.56,0.16}{{#1}}}
    \newcommand{\InformationTok}[1]{\textcolor[rgb]{0.38,0.63,0.69}{\textbf{\textit{{#1}}}}}
    \newcommand{\WarningTok}[1]{\textcolor[rgb]{0.38,0.63,0.69}{\textbf{\textit{{#1}}}}}
    
    
    % Define a nice break command that doesn't care if a line doesn't already
    % exist.
    \def\br{\hspace*{\fill} \\* }
    % Math Jax compatibility definitions
    \def\gt{>}
    \def\lt{<}
    \let\Oldtex\TeX
    \let\Oldlatex\LaTeX
    \renewcommand{\TeX}{\textrm{\Oldtex}}
    \renewcommand{\LaTeX}{\textrm{\Oldlatex}}
    % Document parameters
    % Document title
    \title{TFG}
    
    
    
    
    
% Pygments definitions
\makeatletter
\def\PY@reset{\let\PY@it=\relax \let\PY@bf=\relax%
    \let\PY@ul=\relax \let\PY@tc=\relax%
    \let\PY@bc=\relax \let\PY@ff=\relax}
\def\PY@tok#1{\csname PY@tok@#1\endcsname}
\def\PY@toks#1+{\ifx\relax#1\empty\else%
    \PY@tok{#1}\expandafter\PY@toks\fi}
\def\PY@do#1{\PY@bc{\PY@tc{\PY@ul{%
    \PY@it{\PY@bf{\PY@ff{#1}}}}}}}
\def\PY#1#2{\PY@reset\PY@toks#1+\relax+\PY@do{#2}}

\expandafter\def\csname PY@tok@w\endcsname{\def\PY@tc##1{\textcolor[rgb]{0.73,0.73,0.73}{##1}}}
\expandafter\def\csname PY@tok@c\endcsname{\let\PY@it=\textit\def\PY@tc##1{\textcolor[rgb]{0.25,0.50,0.50}{##1}}}
\expandafter\def\csname PY@tok@cp\endcsname{\def\PY@tc##1{\textcolor[rgb]{0.74,0.48,0.00}{##1}}}
\expandafter\def\csname PY@tok@k\endcsname{\let\PY@bf=\textbf\def\PY@tc##1{\textcolor[rgb]{0.00,0.50,0.00}{##1}}}
\expandafter\def\csname PY@tok@kp\endcsname{\def\PY@tc##1{\textcolor[rgb]{0.00,0.50,0.00}{##1}}}
\expandafter\def\csname PY@tok@kt\endcsname{\def\PY@tc##1{\textcolor[rgb]{0.69,0.00,0.25}{##1}}}
\expandafter\def\csname PY@tok@o\endcsname{\def\PY@tc##1{\textcolor[rgb]{0.40,0.40,0.40}{##1}}}
\expandafter\def\csname PY@tok@ow\endcsname{\let\PY@bf=\textbf\def\PY@tc##1{\textcolor[rgb]{0.67,0.13,1.00}{##1}}}
\expandafter\def\csname PY@tok@nb\endcsname{\def\PY@tc##1{\textcolor[rgb]{0.00,0.50,0.00}{##1}}}
\expandafter\def\csname PY@tok@nf\endcsname{\def\PY@tc##1{\textcolor[rgb]{0.00,0.00,1.00}{##1}}}
\expandafter\def\csname PY@tok@nc\endcsname{\let\PY@bf=\textbf\def\PY@tc##1{\textcolor[rgb]{0.00,0.00,1.00}{##1}}}
\expandafter\def\csname PY@tok@nn\endcsname{\let\PY@bf=\textbf\def\PY@tc##1{\textcolor[rgb]{0.00,0.00,1.00}{##1}}}
\expandafter\def\csname PY@tok@ne\endcsname{\let\PY@bf=\textbf\def\PY@tc##1{\textcolor[rgb]{0.82,0.25,0.23}{##1}}}
\expandafter\def\csname PY@tok@nv\endcsname{\def\PY@tc##1{\textcolor[rgb]{0.10,0.09,0.49}{##1}}}
\expandafter\def\csname PY@tok@no\endcsname{\def\PY@tc##1{\textcolor[rgb]{0.53,0.00,0.00}{##1}}}
\expandafter\def\csname PY@tok@nl\endcsname{\def\PY@tc##1{\textcolor[rgb]{0.63,0.63,0.00}{##1}}}
\expandafter\def\csname PY@tok@ni\endcsname{\let\PY@bf=\textbf\def\PY@tc##1{\textcolor[rgb]{0.60,0.60,0.60}{##1}}}
\expandafter\def\csname PY@tok@na\endcsname{\def\PY@tc##1{\textcolor[rgb]{0.49,0.56,0.16}{##1}}}
\expandafter\def\csname PY@tok@nt\endcsname{\let\PY@bf=\textbf\def\PY@tc##1{\textcolor[rgb]{0.00,0.50,0.00}{##1}}}
\expandafter\def\csname PY@tok@nd\endcsname{\def\PY@tc##1{\textcolor[rgb]{0.67,0.13,1.00}{##1}}}
\expandafter\def\csname PY@tok@s\endcsname{\def\PY@tc##1{\textcolor[rgb]{0.73,0.13,0.13}{##1}}}
\expandafter\def\csname PY@tok@sd\endcsname{\let\PY@it=\textit\def\PY@tc##1{\textcolor[rgb]{0.73,0.13,0.13}{##1}}}
\expandafter\def\csname PY@tok@si\endcsname{\let\PY@bf=\textbf\def\PY@tc##1{\textcolor[rgb]{0.73,0.40,0.53}{##1}}}
\expandafter\def\csname PY@tok@se\endcsname{\let\PY@bf=\textbf\def\PY@tc##1{\textcolor[rgb]{0.73,0.40,0.13}{##1}}}
\expandafter\def\csname PY@tok@sr\endcsname{\def\PY@tc##1{\textcolor[rgb]{0.73,0.40,0.53}{##1}}}
\expandafter\def\csname PY@tok@ss\endcsname{\def\PY@tc##1{\textcolor[rgb]{0.10,0.09,0.49}{##1}}}
\expandafter\def\csname PY@tok@sx\endcsname{\def\PY@tc##1{\textcolor[rgb]{0.00,0.50,0.00}{##1}}}
\expandafter\def\csname PY@tok@m\endcsname{\def\PY@tc##1{\textcolor[rgb]{0.40,0.40,0.40}{##1}}}
\expandafter\def\csname PY@tok@gh\endcsname{\let\PY@bf=\textbf\def\PY@tc##1{\textcolor[rgb]{0.00,0.00,0.50}{##1}}}
\expandafter\def\csname PY@tok@gu\endcsname{\let\PY@bf=\textbf\def\PY@tc##1{\textcolor[rgb]{0.50,0.00,0.50}{##1}}}
\expandafter\def\csname PY@tok@gd\endcsname{\def\PY@tc##1{\textcolor[rgb]{0.63,0.00,0.00}{##1}}}
\expandafter\def\csname PY@tok@gi\endcsname{\def\PY@tc##1{\textcolor[rgb]{0.00,0.63,0.00}{##1}}}
\expandafter\def\csname PY@tok@gr\endcsname{\def\PY@tc##1{\textcolor[rgb]{1.00,0.00,0.00}{##1}}}
\expandafter\def\csname PY@tok@ge\endcsname{\let\PY@it=\textit}
\expandafter\def\csname PY@tok@gs\endcsname{\let\PY@bf=\textbf}
\expandafter\def\csname PY@tok@gp\endcsname{\let\PY@bf=\textbf\def\PY@tc##1{\textcolor[rgb]{0.00,0.00,0.50}{##1}}}
\expandafter\def\csname PY@tok@go\endcsname{\def\PY@tc##1{\textcolor[rgb]{0.53,0.53,0.53}{##1}}}
\expandafter\def\csname PY@tok@gt\endcsname{\def\PY@tc##1{\textcolor[rgb]{0.00,0.27,0.87}{##1}}}
\expandafter\def\csname PY@tok@err\endcsname{\def\PY@bc##1{\setlength{\fboxsep}{0pt}\fcolorbox[rgb]{1.00,0.00,0.00}{1,1,1}{\strut ##1}}}
\expandafter\def\csname PY@tok@kc\endcsname{\let\PY@bf=\textbf\def\PY@tc##1{\textcolor[rgb]{0.00,0.50,0.00}{##1}}}
\expandafter\def\csname PY@tok@kd\endcsname{\let\PY@bf=\textbf\def\PY@tc##1{\textcolor[rgb]{0.00,0.50,0.00}{##1}}}
\expandafter\def\csname PY@tok@kn\endcsname{\let\PY@bf=\textbf\def\PY@tc##1{\textcolor[rgb]{0.00,0.50,0.00}{##1}}}
\expandafter\def\csname PY@tok@kr\endcsname{\let\PY@bf=\textbf\def\PY@tc##1{\textcolor[rgb]{0.00,0.50,0.00}{##1}}}
\expandafter\def\csname PY@tok@bp\endcsname{\def\PY@tc##1{\textcolor[rgb]{0.00,0.50,0.00}{##1}}}
\expandafter\def\csname PY@tok@fm\endcsname{\def\PY@tc##1{\textcolor[rgb]{0.00,0.00,1.00}{##1}}}
\expandafter\def\csname PY@tok@vc\endcsname{\def\PY@tc##1{\textcolor[rgb]{0.10,0.09,0.49}{##1}}}
\expandafter\def\csname PY@tok@vg\endcsname{\def\PY@tc##1{\textcolor[rgb]{0.10,0.09,0.49}{##1}}}
\expandafter\def\csname PY@tok@vi\endcsname{\def\PY@tc##1{\textcolor[rgb]{0.10,0.09,0.49}{##1}}}
\expandafter\def\csname PY@tok@vm\endcsname{\def\PY@tc##1{\textcolor[rgb]{0.10,0.09,0.49}{##1}}}
\expandafter\def\csname PY@tok@sa\endcsname{\def\PY@tc##1{\textcolor[rgb]{0.73,0.13,0.13}{##1}}}
\expandafter\def\csname PY@tok@sb\endcsname{\def\PY@tc##1{\textcolor[rgb]{0.73,0.13,0.13}{##1}}}
\expandafter\def\csname PY@tok@sc\endcsname{\def\PY@tc##1{\textcolor[rgb]{0.73,0.13,0.13}{##1}}}
\expandafter\def\csname PY@tok@dl\endcsname{\def\PY@tc##1{\textcolor[rgb]{0.73,0.13,0.13}{##1}}}
\expandafter\def\csname PY@tok@s2\endcsname{\def\PY@tc##1{\textcolor[rgb]{0.73,0.13,0.13}{##1}}}
\expandafter\def\csname PY@tok@sh\endcsname{\def\PY@tc##1{\textcolor[rgb]{0.73,0.13,0.13}{##1}}}
\expandafter\def\csname PY@tok@s1\endcsname{\def\PY@tc##1{\textcolor[rgb]{0.73,0.13,0.13}{##1}}}
\expandafter\def\csname PY@tok@mb\endcsname{\def\PY@tc##1{\textcolor[rgb]{0.40,0.40,0.40}{##1}}}
\expandafter\def\csname PY@tok@mf\endcsname{\def\PY@tc##1{\textcolor[rgb]{0.40,0.40,0.40}{##1}}}
\expandafter\def\csname PY@tok@mh\endcsname{\def\PY@tc##1{\textcolor[rgb]{0.40,0.40,0.40}{##1}}}
\expandafter\def\csname PY@tok@mi\endcsname{\def\PY@tc##1{\textcolor[rgb]{0.40,0.40,0.40}{##1}}}
\expandafter\def\csname PY@tok@il\endcsname{\def\PY@tc##1{\textcolor[rgb]{0.40,0.40,0.40}{##1}}}
\expandafter\def\csname PY@tok@mo\endcsname{\def\PY@tc##1{\textcolor[rgb]{0.40,0.40,0.40}{##1}}}
\expandafter\def\csname PY@tok@ch\endcsname{\let\PY@it=\textit\def\PY@tc##1{\textcolor[rgb]{0.25,0.50,0.50}{##1}}}
\expandafter\def\csname PY@tok@cm\endcsname{\let\PY@it=\textit\def\PY@tc##1{\textcolor[rgb]{0.25,0.50,0.50}{##1}}}
\expandafter\def\csname PY@tok@cpf\endcsname{\let\PY@it=\textit\def\PY@tc##1{\textcolor[rgb]{0.25,0.50,0.50}{##1}}}
\expandafter\def\csname PY@tok@c1\endcsname{\let\PY@it=\textit\def\PY@tc##1{\textcolor[rgb]{0.25,0.50,0.50}{##1}}}
\expandafter\def\csname PY@tok@cs\endcsname{\let\PY@it=\textit\def\PY@tc##1{\textcolor[rgb]{0.25,0.50,0.50}{##1}}}

\def\PYZbs{\char`\\}
\def\PYZus{\char`\_}
\def\PYZob{\char`\{}
\def\PYZcb{\char`\}}
\def\PYZca{\char`\^}
\def\PYZam{\char`\&}
\def\PYZlt{\char`\<}
\def\PYZgt{\char`\>}
\def\PYZsh{\char`\#}
\def\PYZpc{\char`\%}
\def\PYZdl{\char`\$}
\def\PYZhy{\char`\-}
\def\PYZsq{\char`\'}
\def\PYZdq{\char`\"}
\def\PYZti{\char`\~}
% for compatibility with earlier versions
\def\PYZat{@}
\def\PYZlb{[}
\def\PYZrb{]}
\makeatother


    % For linebreaks inside Verbatim environment from package fancyvrb. 
    \makeatletter
        \newbox\Wrappedcontinuationbox 
        \newbox\Wrappedvisiblespacebox 
        \newcommand*\Wrappedvisiblespace {\textcolor{red}{\textvisiblespace}} 
        \newcommand*\Wrappedcontinuationsymbol {\textcolor{red}{\llap{\tiny$\m@th\hookrightarrow$}}} 
        \newcommand*\Wrappedcontinuationindent {3ex } 
        \newcommand*\Wrappedafterbreak {\kern\Wrappedcontinuationindent\copy\Wrappedcontinuationbox} 
        % Take advantage of the already applied Pygments mark-up to insert 
        % potential linebreaks for TeX processing. 
        %        {, <, #, %, $, ' and ": go to next line. 
        %        _, }, ^, &, >, - and ~: stay at end of broken line. 
        % Use of \textquotesingle for straight quote. 
        \newcommand*\Wrappedbreaksatspecials {% 
            \def\PYGZus{\discretionary{\char`\_}{\Wrappedafterbreak}{\char`\_}}% 
            \def\PYGZob{\discretionary{}{\Wrappedafterbreak\char`\{}{\char`\{}}% 
            \def\PYGZcb{\discretionary{\char`\}}{\Wrappedafterbreak}{\char`\}}}% 
            \def\PYGZca{\discretionary{\char`\^}{\Wrappedafterbreak}{\char`\^}}% 
            \def\PYGZam{\discretionary{\char`\&}{\Wrappedafterbreak}{\char`\&}}% 
            \def\PYGZlt{\discretionary{}{\Wrappedafterbreak\char`\<}{\char`\<}}% 
            \def\PYGZgt{\discretionary{\char`\>}{\Wrappedafterbreak}{\char`\>}}% 
            \def\PYGZsh{\discretionary{}{\Wrappedafterbreak\char`\#}{\char`\#}}% 
            \def\PYGZpc{\discretionary{}{\Wrappedafterbreak\char`\%}{\char`\%}}% 
            \def\PYGZdl{\discretionary{}{\Wrappedafterbreak\char`\$}{\char`\$}}% 
            \def\PYGZhy{\discretionary{\char`\-}{\Wrappedafterbreak}{\char`\-}}% 
            \def\PYGZsq{\discretionary{}{\Wrappedafterbreak\textquotesingle}{\textquotesingle}}% 
            \def\PYGZdq{\discretionary{}{\Wrappedafterbreak\char`\"}{\char`\"}}% 
            \def\PYGZti{\discretionary{\char`\~}{\Wrappedafterbreak}{\char`\~}}% 
        } 
        % Some characters . , ; ? ! / are not pygmentized. 
        % This macro makes them "active" and they will insert potential linebreaks 
        \newcommand*\Wrappedbreaksatpunct {% 
            \lccode`\~`\.\lowercase{\def~}{\discretionary{\hbox{\char`\.}}{\Wrappedafterbreak}{\hbox{\char`\.}}}% 
            \lccode`\~`\,\lowercase{\def~}{\discretionary{\hbox{\char`\,}}{\Wrappedafterbreak}{\hbox{\char`\,}}}% 
            \lccode`\~`\;\lowercase{\def~}{\discretionary{\hbox{\char`\;}}{\Wrappedafterbreak}{\hbox{\char`\;}}}% 
            \lccode`\~`\:\lowercase{\def~}{\discretionary{\hbox{\char`\:}}{\Wrappedafterbreak}{\hbox{\char`\:}}}% 
            \lccode`\~`\?\lowercase{\def~}{\discretionary{\hbox{\char`\?}}{\Wrappedafterbreak}{\hbox{\char`\?}}}% 
            \lccode`\~`\!\lowercase{\def~}{\discretionary{\hbox{\char`\!}}{\Wrappedafterbreak}{\hbox{\char`\!}}}% 
            \lccode`\~`\/\lowercase{\def~}{\discretionary{\hbox{\char`\/}}{\Wrappedafterbreak}{\hbox{\char`\/}}}% 
            \catcode`\.\active
            \catcode`\,\active 
            \catcode`\;\active
            \catcode`\:\active
            \catcode`\?\active
            \catcode`\!\active
            \catcode`\/\active 
            \lccode`\~`\~ 	
        }
    \makeatother

    \let\OriginalVerbatim=\Verbatim
    \makeatletter
    \renewcommand{\Verbatim}[1][1]{%
        %\parskip\z@skip
        \sbox\Wrappedcontinuationbox {\Wrappedcontinuationsymbol}%
        \sbox\Wrappedvisiblespacebox {\FV@SetupFont\Wrappedvisiblespace}%
        \def\FancyVerbFormatLine ##1{\hsize\linewidth
            \vtop{\raggedright\hyphenpenalty\z@\exhyphenpenalty\z@
                \doublehyphendemerits\z@\finalhyphendemerits\z@
                \strut ##1\strut}%
        }%
        % If the linebreak is at a space, the latter will be displayed as visible
        % space at end of first line, and a continuation symbol starts next line.
        % Stretch/shrink are however usually zero for typewriter font.
        \def\FV@Space {%
            \nobreak\hskip\z@ plus\fontdimen3\font minus\fontdimen4\font
            \discretionary{\copy\Wrappedvisiblespacebox}{\Wrappedafterbreak}
            {\kern\fontdimen2\font}%
        }%
        
        % Allow breaks at special characters using \PYG... macros.
        \Wrappedbreaksatspecials
        % Breaks at punctuation characters . , ; ? ! and / need catcode=\active 	
        \OriginalVerbatim[#1,codes*=\Wrappedbreaksatpunct]%
    }
    \makeatother

    % Exact colors from NB
    \definecolor{incolor}{HTML}{303F9F}
    \definecolor{outcolor}{HTML}{D84315}
    \definecolor{cellborder}{HTML}{CFCFCF}
    \definecolor{cellbackground}{HTML}{F7F7F7}
    
    % prompt
    \makeatletter
    \newcommand{\boxspacing}{\kern\kvtcb@left@rule\kern\kvtcb@boxsep}
    \makeatother
    \newcommand{\prompt}[4]{
        \ttfamily\llap{{\color{#2}[#3]:\hspace{3pt}#4}}\vspace{-\baselineskip}
    }
    

    
    % Prevent overflowing lines due to hard-to-break entities
    \sloppy 
    % Setup hyperref package
    \hypersetup{
      breaklinks=true,  % so long urls are correctly broken across lines
      colorlinks=true,
      urlcolor=urlcolor,
      linkcolor=linkcolor,
      citecolor=citecolor,
      }
    % Slightly bigger margins than the latex defaults
    
    \geometry{verbose,tmargin=1in,bmargin=1in,lmargin=1in,rmargin=1in}
    
    

\begin{document}
    
    \maketitle
    
    

    
    \begin{tcolorbox}[breakable, size=fbox, boxrule=1pt, pad at break*=1mm,colback=cellbackground, colframe=cellborder]
\prompt{In}{incolor}{1}{\boxspacing}
\begin{Verbatim}[commandchars=\\\{\}]
\PY{o}{\PYZpc{}}\PY{k}{config} InlineBackend.figure\PYZus{}format = \PYZsq{}retina\PYZsq{}  \PYZsh{} Configuración 144 píxeles para imágenes
\end{Verbatim}
\end{tcolorbox}

    \begin{itemize}
\tightlist
\item
  El texto debe estar justificado.
\item
  Todas las figuras y tablas deben tener su pie indicando la numeración
  y descripción en la parte inferior.
\item
  Se debe incluir un índice de capítulos y de símbolos (variables y
  acrónimos).
\item
  Referencias bibliográficas con un número entre corchetes. No te
  olvides de referenciar cualquier modelo o dato que saques de otros
  documentos o webs.
\item
  Ecuaciones numeradas.
\item
  En el pie de página numeración y si quieres puedes poner un encabezado
  con el título del proyecto. Aquí ya no hay formato específico.
\end{itemize}

INDICE INTRODUCCIÓN - OBJETIVO - ESTRUCTURA DE LA MEMORIA - MODELADO DE
CONCENTRADORES CILINDROPARABOLICOS

DESCRIPCIÓN DE LOS MODELOS TEORICOS UTILIZADO - ECUACIONES DEL MODELO -
MODELO DE 4 ORDEN - MODELO DE PRIMER ORDEN - MODELO SIMPLIFICADO -
LIMITACIONES DE LOS MODELOS

MODELADO DEL SISTEMA - PARADIGMA DE LA POO APLICADO A LA INGENIERÍA -
DEFINICIÓN DE CLASES PARA MODELAR EL CAMPO SOLAR - CLASES
CORRESPONDIENTES A LOS SISTEMAS FÍSICOS - HERRAMIENTAS UTILIZADAS PARA
EL TFG - LIBRERIAS - ACCESO A LA DOCUMENTACIÓN

DESCRIPCIÓN DEL CÓDIGO - PRIMEROS RESULTADOS Y VALIDACIÓN - ALGUNOS
EJEMPLOS

SIMULACIÓN DE UN CAMPO SOLAR - DESCRIPCIÓN DE LA CONFIGURACIÓN DEL CAMPO
- RESULTADOS DE LA SIMULACIÓN SEGÚN CASOS CAMBIO GRADO MODELO, CAMBIO
TAMAÑO HCE, CAMBIO PRECISION, CAMBIO FICHERO HORARIO (MES TIPO U
HORARIO)

CONCLUSIONES Y TRABAJO FUTURO

INDICE DE FIGURAS INDICE DE TABLAS LISTA DE SÍMBOLOS REFERENCIAS

\hypertarget{introducciuxf3n}{%
\subsection{Introducción}\label{introducciuxf3n}}

\hypertarget{objetivo}{%
\subsubsection{Objetivo}\label{objetivo}}

El propósito principal de este Trabajo Final de Grado (TFG) es
profundizar en el conocimiento del funcionamiento de un campo solar de
concentradores cilindroparabólicos (CCP) mediante el desarrollo de una
herramienta de simulación programada en Python 3. Para ello, se ha
partido del modelo teórico desarrolado en su Tesis Doctoral por el
profesor de la Universidad Nacional de Educación a Distancia (UNED), el
Dr.~Rubén Barbero Fresno y se ha empleado una metodología basada en el
paradigma de la programación orientada a objetos (POO). El desarrolo de
este trabajo ha supuesto un reto personal por la necesidad de adquirir
habilidades en el manejo de diferentes herramientas antes desconocidas
para mí, como el lenguaje de programación Python 3 y sus diferentes
librerías para el cálculo científico (Numpy) y el tratamiento de datos
(Pandas). Finalmente, todo el código se encuentra publicado y accesible
a través de GitHub y se puede interactuar con la versión final del
código de simulación a través de un Notebook Jupyther.

Los resutaldos obtenidos son compatibles con los datos de generación de
un campo solar real dentro de las limitaciones que los diferentes modos
de operación y eventos impredecibles (paradas de planta para
mantenimiento, disparos por avería, etc\ldots{}) introducen en el
proceso. En este proyecto solo se aborda la simulación del sistema del
campo solar, el único sistema dentro del alcance del modelo teórico de
partida. Para la simulación de planta sería necesario el desarrollo de
modelos para gran número de sistemas como por ejemplo, los generadores
de vapor, almacenamiento térmico, turbina, sistemas auxiliares de
bombeo, sistemas de tratamiento de agua, torre de refrigeración,
etc\ldots{} No obstante, la metodología seguida permite que en el futuro
este proyecto pueda ser ampliado de forma sistemática con la
incoroporación de nuevas clases de objetos que aprovechen los métodos de
entrada y salida ya programados para interactuar con ellos. En todo
caso, se han empleado valores de rendimientos estimados para los
principales subsistemas de planta con el fin de ofrecer una estimación
de la energía eléctrica finalmente vertida a la red.

\hypertarget{estructura-de-la-memoria}{%
\subsubsection{Estructura de la
memoria}\label{estructura-de-la-memoria}}

Esta memoria se estructura en 6 capítulos y un anexo con el código
fuente.\\
* En el primer capítulo, del que forma parte esta introducción se
describen los objetivos del TFG y se ofrece una introducción en la que
se ofrecen algunos apuntes sobre la simulación de CCP, comentando
algunas de las herramientas disponibles en la utilidad y su modo de
funcionamiento. * El segúndo capítulo describe el modelo teórico de
partida y se detallan las ecuaciones que describen los sistemas que más
adelante serán modelados. * El tercer capítulo aborda el modelado de los
diferentes sistemas necesarios para la caracterización del campo solar.
* En el cuarto capítulo se presenta el código desde un punto de vista
funcional y se muestran algunos ejemplos de interés para la
caracterizacion de CCP * En el quinto capítulo se describe el proceso a
segir para definir la confiuración del campo solar de tal forma que
pueda ser interpretada por el código. Se acomete la simulación de una
planta real y se comparan los resultados de la simulación con los datos
disponibles de dicha planta. * El sexto capítulo se dedica a presentar
conclusiones y se realizan algunas propuestas de desarrollo futuro.

\hypertarget{modelado-de-concentradores-cilindroparabuxf3licos}{%
\subsubsection{Modelado de concentradores
cilindroparabólicos}\label{modelado-de-concentradores-cilindroparabuxf3licos}}

Un concentrador cilindroparabólico (CCP) consiste en un reflector
parabólico cuya geometría se obtiene por la traslación de una parábola a
lo largo de un eje que contiene su foco. Dentro de los diferentes
sistemas de concentración solar pertenece al grupo de los concentradores
lineales, al igual que los sistemas de concentración tipo Fresnel y al
contrario que los sistemas de concentración de torre central o de discos
parabólicos, en cuyo caso estaríamos hablando de sistemas de
concentración puntuales.

A lo largo del eje focal del concentrador se instala una conducción por
la que circula un fluido caloportador o transmisor del calor (HTF por
sus siglas en ingles, Heat Transfer Fluid). Esta conducción está
compuesta en realidad por una serie de elentos tubulares denominados
Heat Collector Element, HCE. Los HCE consisten en tubo de acero con una
envolvente de vidrio de tal forma que en el proceso de fabricación se ha
dejado extraido el aire que queda entre ambos (region anular o annulus).
De esta forma se reducen las pérdidas de calor por convección a través
de la región anular. La soldadura vidrio-metal y unos elementos
denominados getters que absorben, hasta cierto punto, algunas moléculas
que puedan filtrarse a la región anular durnte la vida de operacion del
HCE permiten que éste cuente con pérdidas reducias de calor mientras no
se produzca la rotura del vidrio o la saturación de dichos getters.

Para que se produzca la concentración de la radiación solar incidente
ésta debe ser perpendicular al eje que pasa por el foco y la base de la
parábola. La primera consecuenca es que solo puede aprovecharse
plenamente la componente normal de la radiación solar incidente (DNI).
Dado que el Sol varía su posición relativa al concentrador
continuamente, el conjunto reflector-tubo absorbedor está montado sobre
una estructura que pueda girar sobre un eje con el fin de segir la
trayectoria solar a lo largo del día. Salvo instalaciones especiales en
laboratorios, no se emplean seguidores a dos ejes por la dificultad
técnica que acarrearía desarrollar este tipo de estructuras. Las plantas
solares cuyo objetivo es maximizar el vertido anual de energía eléctrica
a la red cuentan con una orientación de su eje de segumiento Norte-Sur,
lo cual lleva a que exista una importante diferencia entre la generación
en los meses de verano y los meses de invierno, siendo mayor en los
primeros. Si lo que se persigue es obtener una producción más estable a
lo largo del año, la orientación más adecuda del eje sería Este-Oeste.

    \hypertarget{descripciuxf3n-de-los-modelos-teuxf3ricos-utilizados}{%
\subsection{Descripción de los modelos teóricos
utilizados}\label{descripciuxf3n-de-los-modelos-teuxf3ricos-utilizados}}

\hypertarget{modelo-de-rendimiento-tuxe9rmico-de-4uxba-orden-para-caracterizaciuxf3n-de-un-sistema-de-captaciuxf3n-solar}{%
\subsubsection{Modelo de rendimiento térmico de 4º Orden para
caracterización de un sistema de captación
solar}\label{modelo-de-rendimiento-tuxe9rmico-de-4uxba-orden-para-caracterizaciuxf3n-de-un-sistema-de-captaciuxf3n-solar}}

El modelo de partida desarrollado en \cite{1022085/7TD8VTGL} tiene un
caracter general que permite que sea aplicado a receptores térmicos de
radiación solar de cualquier tecnología, tanto para concentradores
cilindroparabólicos como para conentradores lineales Fresnel o
receptores de torre central. En este trabajo nos centramos en los
aspectos relativos a los CCP y a continuación revisaremos las
características del modelo para este tipo concreto de receptores.

El receptor considerado inicialmente consiste en un tubo desnudo de
diámetro \(D_{ro}\) (m) y longitud \(L\) (m). A lo largo del desarrollo
del modelo se realizan ciertas aproximaciones para las que se aportan
justificaciones que no repetiremos ahora, pero que pueden encontrarse en
el texto original. Como primera aproximación se considera que el
receptor abosorbe radiación de manera uniforme a través de toda su
superficie. Igualmente se desprecia la transerencia de calor en la
dirección axial.

El esquema propuesto parte del balance energético del receptor,
ec.\(\eqref{eq:balance_receptor}\)

\begin{equation}
    \dot{q}^{"}_{perd}(x) = \dot{q}^{"}_{abs}-\dot{q}^{"}_{u}(x) \label{eq:balance_receptor}
\end{equation}

donde \(\dot{q}^{"}_{perd}\) es la energía perdida para una sección a
una distancia \(x\) de la entrada al receptor, \(\dot{q}^{"}_{abs}\) es
la radiación absorbida y \(\dot{q}^{"}_{u}\) es la energía útil . En la
Fig.\(\ref{fig:receptormodelo}\) se representa el modelo y se representa
mediante flechas el sentido de flujo de la energía.

\begin{figure}
\includegraphics[scale=0.8]{images/receptor_para_modelo.png}
\centering
\caption{Esquema del receptor empleado para el modelo. Fuente:(Barbero Fresno 2018)} 
\label{fig:receptormodelo}
\end{figure}

\begin{figure}    
\includegraphics[scale=0.8]{}
\centering
\caption{Esquema del receptor empleado para el modelo. 
Fuente:(Barbero Fresno 2018)} 
\label{fig:receptor}
\end{figure}

La expresión para el cálculo de la radiación absorbida se muestra en la
ec.\(\eqref{eq:qabs}\).

\begin{equation}
    \dot{q}^{"}_{abs}= \eta_{opt}(\theta) \cdot Cg \cdot DNI \cdot \eta_{sombras} \cdot \eta_{bordes} \label{eq:qabs}
\end{equation}

El rendimiento óptico \(\eta_{opt}\), las pérdidas geométricas
\(\eta_{bordes}\) y las pérdias por sombras, \(\eta_{sombras}\) son
valores conocidos o que pueden calcularse para cada momento en función
de la geometría del receptor y la disposicón de los concentradores en el
campo solar. \(Cg\) es el factor de concentración y también es conocido
a partir de la geometría del conjunto concentrador-receptor. Finalmente,
\(DNI\) es la radiación normal incidente. La ec.\(\eqref{eq:qu}\)
permite hallar al calor tranferido desde el tubo abosrbedor a
temperatura \(T_{ro}\), al fluído térmico a temperatura \(T_{f}\):

\begin{equation}
    \dot{q}^{"}_{u}(x)= U_{rec} \cdot [T_{ro}(x)-T_{f}(x)] \label{eq:qu}
\end{equation}

donde \(U_{rec}\) es el ooeficiente global de transferencia de calor
hacia el interior, cuya expresión se muestra en la
ec.\(\eqref{eq:urec}\):

\begin{equation}
    U_{rec} = \frac{1}{\frac{1}{h_{int}} + \frac{D_{ro}\cdot(\frac{D_{ro}}{D_{ri})})}{2\cdot k_{rec}}} \label{eq:urec}
\end{equation}

Como aproximación se considera que \(U_{rec}\) es constate a lo largo de
la longitud del tubo \((W/m^{2}\cdot K)\). \(D_{ro}\) y \(D_{ri}\) son
el diámetro exterior e interior \((m)\) respectivamente del tubo
absorbedor. \(h_{int}\) es el coeficiente de transferencia de calor
convectivo hacia el interior \((W/m^{2}\cdot K)\) y \(k_{rec}\) es
conductivdad del material del receptor, en \((W/m\cdot K)\)

Para las pérdidas se calculan mediante la ec.\(\eqref{eq:qperd}\) de
calor se tendrá en cuenta un término radiativo, con temperaturas de 4o
grado y otro convectivo de 1er grado.

\begin{equation}
    \dot{q}^{"}_{perd}(x)= \sigma \cdot \varepsilon_{ext} \cdot (T_{ro}^{4}(x)-T_{ext}^{4}) + h_{hext} \cdot (T_{ro}(x)-T_{ext}) \label{eq:qperd}
\end{equation}

donde \(\sigma\) es la constante de Stefan-Boltzmann (5.67x10-8 W/m2K4),
\(\varepsilon_{ext}\) es la emisividad del tratamiento superficial
exterior y \(h_{hext}\) es el coeficiente de convección exterior. Estas
dos últimas constantes son características de cada receptor y variables,
por ejemplo, en función de las condiciones de degradación del
recubrimiento selectivo o del viento exterior. Deben ser halladas
experimentalmente en laboratorio.

La última ecuación necesaria es la ec.\(\eqref{eq:deltaT}\), donde se
calcula el incremento de temperatura que experimenta el fluido
considerando despreciables los cambios en energía cinética y un calor
específico constante, \(c_{p}\) (J/Kg·K). Denominando \(T_{f}(x)\) a la
tempertura del fluido en la sección a distancia \(x\) de la entrada y
\(T_{fe}\) a la temperatura del fluido a la entrada, tenemos:

\begin{equation}
    \pi \cdot D_{ro} \cdot x \cdot \dot{q}^{"}_{abs} \cdot \eta(x)= \dot{m} \cdot c_{p} \cdot (T_{f}(x)-T_{fe}) \label{eq:deltaT}
\end{equation}\\

En esta última ecuación aparece el rendimiento integral hasta una
sección a una distancia \(x\) de la entrada, \(\eta(x)\). Finalmente, a
partir del rendimiento local, dado por la
ec.\(\eqref{eq:rendimientolocal}\) podemos calcular el rendimiento
integral mediante la ec.\(\eqref{eq:rendimientointegral}\).

\begin{equation}
    \eta_{x}(x) = \frac{\dot{q}^{"}_{u}(x)}{\dot{q}^{"}_{abs}} \label{eq:rendimientolocal}
\end{equation}

\begin{equation}
    \eta(x) = \frac{\int_{0}^{x}\eta_{x}(x)dx}{\int_{0}^{x}dx} \label{eq:rendimientointegral}
\end{equation}

donde desarrollando \(\eta_{x}(x)\) según la ec.\(\eqref{eq:etax}\):

\begin{equation}
    \eta_{x}(x) = \eta(x) + \eta'(x)\cdot x  \label{eq:etax}
\end{equation}

y normalizando la distancia a la unidad con la variable adimensional
\(x^{*}=x/L\), obtenemos la ecuación integral
ec.\(\eqref{eq:rendimientonormalizado}\):

\begin{equation}
    \eta(x^{*}) = 1 - \frac{\int_{0}^{x^{*}} \dot{q}^{"}_{perd}(dx^{*})\cdot dx^{*}}{\dot{q}^{"}_{abs}\cdot dx^{*}} \label{eq:rendimientonormalizado}
\end{equation}

La resolución de esta ecuación requiere un largo desarrollo en el que se
introducen nuevos factores característicos del sistema y que puede
encontrarse en la obra de referencia, por lo que la omitiremos aquí.
Pese a la complejidad de la expresión final obtenida, que dificulta
extraer conclusiones de manera directa, el modelo incorpora todos los
parámetros característicos del sistema y lo hace manteniendo su sentido
físico. A partir de la solución se puede obtener una expresión para el
modelo local (rendimiento en una sección determinada del absorbedor) y
una expresión para el modelo de colector completo, es decir, un
rendimiento integral a lo largo de todo el absorbedor. Esta última es la
que nos interesa. A continuación se presenta la ecuación del Modelo de
4º Orden completo del colector en la ec.\(\eqref{eq:modelocompleto}\) y
sucesivamente las ecuaciones que definen sus parámetros:

\begin{equation}
    \eta(x^{*}) = \frac{\eta_{0} \cdot g'(Z)}{1-g'(Z)} \cdot \frac{1}{NTU \cdot x^{*}} \cdot \left(e^{\frac{1-g'(Z)}{g'(Z)}\cdot NTU \cdot x^{*}} - 1\right) - \frac{\eta_{0}^2}{6} \cdot \frac{g''(Z)}{g'(Z)} \cdot NTU^{2} \cdot x^{*^{2}} - \frac{\eta_{0}^{3}}{24} \cdot \frac{g'''(Z)}{g'(Z)} \cdot NTU^{3} \cdot x^{*^{3}}
    \label{eq:modelocompleto}
\end{equation}

\begin{equation}
    \eta_{0} = 1 - (f_{1} \cdot Z + f_{2} \cdot Z^{2} + f_{3} \cdot Z^{3} + f_{4} \cdot Z^{4})
    \label{eq:rendimiento0}
\end{equation}

\begin{equation}
    Z = \eta_{0} + \frac{1}{f_{0}} 
    \label{eq:zeta}
\end{equation}

\begin{equation}
    g(Z) = -\left(1+\frac{1}{f_{0}}\right)+(1+f_{1})\cdot Z + f_{2}\cdot Z^{2} +  f_{3}\cdot Z^{3} + f_{4}\cdot Z^{4} 
    \label{eq:gdezeta}
\end{equation}

\begin{equation}
    g'(Z) = 1 + f_{1} + 2 \cdot f_{2} \cdot Z + 3 \cdot f_{3} \cdot Z^{2} + 4 \cdot f_{4} \cdot Z^{3} 
    \label{eq:gprimadezeta}
\end{equation}

\begin{equation}
    g''(Z) = 2 \cdot f_{2} \cdot Z + 6 \cdot f_{3} \cdot Z + 12 \cdot f_{4} \cdot Z^{2} 
    \label{eq:g2primadezeta}
\end{equation}

\begin{equation}
    g'''(Z) = 6 \cdot f_{3} + 24 \cdot f_{4} \cdot Z
    \label{eq:g3primadezeta}
\end{equation}

\begin{equation}
    g^{IV}(Z) = 24 \cdot f_{4}
    \label{eq:g4primadezeta}
\end{equation}

\begin{equation}
    f_{1} = \frac{4 \cdot \sigma \cdot \varepsilon_{ext} \cdot T_{ext}^{3} + h_{ext}}{U_{rec}}
    \label{eq:f1}
\end{equation}

\begin{equation}
    f_{2} = 6 \cdot T_{ext}^{2} \cdot \left(\frac{\sigma \cdot \varepsilon_{ext}}{U_{rec}} \right) \cdot \left(\frac{\dot{q}^{"}_{abs}}{U_{rec}}\right) 
    \label{eq:f2}
\end{equation}

\begin{equation}
    f_{3} = 4 \cdot T_{ext} \cdot \left(\frac{\sigma \cdot \varepsilon_{ext}}{U_{rec}} \right) \cdot \left(\frac{\dot{q}^{"}_{abs}}{U_{rec}}\right)^{2} 
    \label{eq:f3}
\end{equation}

\begin{equation}
    f_{4} = \left(\frac{\sigma \cdot \varepsilon_{ext}}{U_{rec}} \right) \cdot \left(\frac{\dot{q}^{"}_{abs}}{U_{rec}}\right) 
    \label{eq:f4}
\end{equation}

Para la resolución de este modelo es preciso conocer previamente
diferentes parámetros, muchos de los cuales pueden obtenerse
directamente de las característias físicas y materiales con los que está
construido el HCE. De especial importancia son \(\varepsilon_{ext}\) y
\(h_{ext}\) pues son dos coeficientes que de forma global vienen a
caracterizar las pérdidas energéticas del receptor. Para obtener las
ecuaciones que los caracterizan se parte de la expresión del coeficiente
global del pérdidas al exterior dada por la ec.\(\eqref{eq:uext}\) de la
siguiente forma:

\begin{equation}
    U_{ext} = h_{hext} + \sigma \cdot \varepsilon_{ext} \cdot \left(T_{ro}^2 + T_{ext}^2 \right) \cdot \left(T_{ro} + T_{ext} \right)
    \label{eq:uext}
\end{equation}

Es necesario realizar ensayos de laboratorio bajo diferentes condiciones
de viento (\(W_{spd}\)) y temperatura exterior (\(T_{ext}\)) para
obtener el flujo de calor de pérdidas y calcular así dos expresiones del
tipo \(\varepsilon_{ext}(T_{ext},W_{spd})\) y
\(h_{ext}(T_{ext},W_{spd})\). Para este trabajo se emplean los valores
obtenidos en \cite{1022085/7TD8VTGL} a partir de
\cite{1022085/2CSZR6MB}, \cite{1022085/Z2N8UMZB},
\cite{1022085/8AF3BT7W} .

A partir de este modelo de 4º Orden se realiza un desarrollo que permite
obtener dos modelos simplificados de colector completo: el Modelo de
Primer Orden y el Modelo Simplificado.

\hypertarget{modelo-de-primer-orden}{%
\subsubsection{Modelo de Primer Orden}\label{modelo-de-primer-orden}}

La ec.\(\eqref{eq:primerorden}\) presenta el Modelo de Primer Orden.
Para llegar a ella resuelve la ec.\(\eqref{eq:rendimiento0}\)
despreciando monomios a partir de segundo grado, con lo que se puede
sustituir el rendimiento a la entrada del absorbedor, \(\eta_{0}\), por
su valor aproximado dado en la ec.\(\eqref{eq:rendimiento0aproximado}\):

\begin{equation}
    \eta(x^{*}) = \left[1-\frac{\dot q''_{crit}}{\dot q''_{abs}}\right] \cdot \frac{1}{NTU_{perd} \cdot x^{*}} \cdot \left(1-e^{-NTU_{perd}\cdot F'_{crit}\cdot x^{*}}\right) 
    \label{eq:primerorden}
\end{equation}

\begin{equation}
    \eta_{0} = F'_{crit} \cdot \left[1-\frac{\dot q''_{crit}}{\dot q''_{abs}}\right] 
    \label{eq:rendimiento0aproximado}
\end{equation}

En esta ecuación se han reagrupado variables en diferentes términos que
cuentan con sentido físico. De este modo, se definen:

\begin{equation}
    \dot q''_{crit} = \sigma \cdot \varepsilon_{ext} \cdot \left(T^{4}_{fe}- T^{4}_{ext}\right)+h_{hext} \cdot \left(T^{4}_{fe}- T^{4}_{ext}\right)
    \label{eq:qcrit}
\end{equation}

\begin{equation}
    U_{crit} = 4 \cdot \sigma \cdot \varepsilon_{ext} \cdot T^{3}_{fe} + h_{hext}
    \label{eq:ucrit}
\end{equation}

\(\dot q''_{crit}\) y \(U_{crit}\) son valores de referencia en el
estudio del comportamiento del colector pues cuando \(\dot q''_{abs}\) y
\(U_{rec}\) se aproximan a ellos el rendimiento del colector se hace
nulo. Por otra parte, \(F'_{crit}\) se asemeja al parámetro empleado en
el modello desarrolado por Hottel y Whillier en \cite{1022085/DVRL97SH}.

\begin{equation}
    F'_{crit} = \frac{1}{\frac{4 \cdot \sigma \cdot \varepsilon_{ext} \cdot T^{3}_{fe}}{U_{rec}} + \frac{h_{ext}}{U_{rec}} +1} = \frac{1}{\frac{U_{crit}}{U_{rec}}+1}
    \label{eq:fcrit}
\end{equation}

El Modelo de Primer Orden presenta la ventaja de que el cálculo de
\(\eta(x^{*})\) es explícito. Más adelante veremos cómo repercute esto
en la reducción del coste computacional. Por otro lado, también nos
ofrece una forma de calcular un valor aproximado del rendimiento a la
entrada del colector, \(\eta_{0}\).

\hypertarget{modelo-simplificado}{%
\subsubsection{Modelo Simplificado}\label{modelo-simplificado}}

Si se desarrolla por Taylor la funcion exponencial del Modelo de Primer
Orden, se trunca por el segundo término y se sustituye
\(\dot q''_{abs}\) por su expresión en función de \(DNI\) se obtiene la
ec.\(\eqref{eq:modelosimplificado}\) para el cálculo del rendimiento
total en el receptor mediante el Modelo Simplificado:

\begin{equation}
    \eta_{T} = F'_{crit} \cdot \left[1 - \frac{\dot q''_{crit}}{\dot q''_{abs}}\right] = \frac{F'_{crit}}{Cg} \cdot \left[Cg \cdot IAM \cdot cos(\theta) \cdot \eta_{opt,pico} \cdot \eta_{sombras} \cdot \eta_{bordes} - \frac{h_{ext}\cdot (\bar{T}_{f}-\bar{T}_{ext})}{DNI} - \frac{\sigma \cdot \varepsilon_{ext}\cdot(\bar{T}^{4}_{f}-\bar{T}^{4}_{ext})}{DNI}\right] 
    \label{eq:modelosimplificado}
\end{equation}

Esta ecuación es más parecida a la encontrada en otros modelos de
diferentes autores, como por ejemplo en
\cite{1022085/DVRL97SH}\cite{1022085/HI2ETXYA}, pero con dependencia de
\(T^{4}\) lo cual tiene mayor sentido físico al esperarse que las
pérdidas radiativas sean dominantes en situaciones de media y alta
concentración.

\hypertarget{aplicabilidad-de-los-modelos}{%
\subsubsection{Aplicabilidad de los
modelos}\label{aplicabilidad-de-los-modelos}}

Aunque en el caso del Modelo de 4º Orden no se ha realizado ninguna
simplificación para la resolución de la ecuación caracterítica, sí que
se han hecho las siguientes consdiraciones que limitan su aplicacion: *
Se ha considerdo que los parametros característicos \(U_{rec}\),
\(\varepsilon_{ext}\), \(h_{ext}\) y \(Cp\) son constantes a lo largo de
toda la longitud del receptor. Se considerará que esto es aceptable para
longitudes inferiores a 100 m tal y como se indica en el desarrollo del
modelo. * Se supone uniformidad del flujo de radiación sobre el tubo
absorbedor. Para tecnología CCP se acepta esta hipótesis. * La
caracterización de los tubos absorbedores empleados en CCP es compatible
con el desarrollo del modelo basada en un tubo desnudo (para los que
posteriormente se emplearán unos coeficientes de trasmisión de calor
adecuados para los tubos con cubierta de vidrio). * La suposición de
fluido incompresible en la que se desprecia el término de pérdida de
carga y de energía cinética sobre el término energético es adecuada para
plantas que operan con aceite térmico dado que los circuitos están
presurizados para mantener en todo momento el fluido en estado líquido y
los caudales de operación tienen un número de Reynolds medio. * El flujo
puede suponerse uniforme en el interior del tubo absorbedor. * Dada la
longitud real del tubo absorbedor en una planta CCP, puede despreciarse
el efecto de transmisión de calor longitudinal.

Según lo visto, el modelo resulta aplicable a la simulación de un campo
solar de concentradores cilindroparabólicos bajo las condiciones
normales de operación. Por otro lado, la simulación se realizará también
para el cálculo con intervalos horarios en los que se supondrá
condiciones estacionarias de planta y se descartarán aquellos periodos
de arranque y parada o cambios abruptos en los que las inercias propias
del sistema y la intervención de los operadores de planta producirían
que el comportamiento instantáneo no se correspondiese con el simulado.

El Modelo de Primer Orden presenta mayores restricciones, especialmente
en los rangos de operación aceptables (no deben ser próximos a los
valores críticos definidos en la ec.\(\eqref{eq:ucrit}\) y
\(\eqref{eq:qcrit}\)), siendo sus resultados algo menos precisos en
general.

El Modelo Simplificado solo es válido además para longitudes de receptor
más reducidas.

    \hypertarget{modelado-del-campo-solar.}{%
\subsection{Modelado del campo solar.}\label{modelado-del-campo-solar.}}

\hypertarget{paradigma-de-la-programaciuxf3n-orientada-a-objetos-para-la-simulaciuxf3n-de-sistemas-fuxedsicos}{%
\subsubsection{Paradigma de la Programación Orientada a Objetos para la
simulación de sistemas
físicos}\label{paradigma-de-la-programaciuxf3n-orientada-a-objetos-para-la-simulaciuxf3n-de-sistemas-fuxedsicos}}

Para la programación del código que realice la simulación del sistema se
ha optado por seguir un paradigma de Programación Orientada a Objetos
(POO) según el cual cada sistema físico se define como un objeto
perteneciente a una Clase con la capacidad de recibir información,
manipularla de acuerdo a unas reglas propias del sistema y devolver
información.

Una de las principales ventajas de esta metodología es la modularidad,
de tal forma que se puede ir desarrollando jerárquica, progresiva e
independientemente cada uno de los sistemas para después
interconectarlos. Posteriormente se puede modificar el comportamiento de
alguno de estos objetos reprogramando la Clase a la que pertenece sin
que esto afecte de forma drástica al resto de objetos del modelo. Es una
técnica escalable y que permite definir diferentes grados de
intervención al usuario final, desde interactuar con cada objeto como si
de una caja negra se tratase hasta modificar el comportamiento del
sistema introduciendo sus propios métodos en las clases.

Por todas estas razones, el modelado mediante POO resulta muy
interesante para la simulación de sistemas en el ámbito de la ingeniería
y ha sido el elegido para el desarrollo del código de este TFG.

\hypertarget{sistemas-fuxedsicos-y-clases-para-el-modelado-del-campo-solar.}{%
\subsubsection{Sistemas físicos y Clases para el modelado del campo
solar.}\label{sistemas-fuxedsicos-y-clases-para-el-modelado-del-campo-solar.}}

En los siguientes apartadoos iremos describiendo el campo solar desde el
punto de vista de su comportamiento físico, los subsistemas que lo
componen y definiremos las Clases que se deben programar para modelar
cada uno de estos subsistemas. Pero en primer lugar introduciremos
alguna terminología.

Se denomina HCE (Heat Collector Element) a cada uno de los tubos
absorbedores de unos 4 m de longitud con envolvente de vidrio propia
que, soldados uno tras otro forman la tubería sobre la que concentra la
radiación solar. Estos tubos vienen ensamblados en grupos de 2 o 3 para
acelerar el proceso de construcción, de tal forma que pueden colocarse
sobre una sección de concentrador denominada SCE (Solar Collector
Element) de unos 12 m de longitud.

\begin{verbatim}
FIGURA HCE
\end{verbatim}

Un conjunto de SCE que se mueven solidariamente con capacidad de
movmiennto independiente se denomina SCA (Solar Collector Assambly). El
tubo absorbedor montado en cada SCA está unido mediante uniones móviles
al tubo absorbedor del siguietne SCA o a las tuberías de entrada y
salida del lazo. El SCA es, por tanto, la unidad mínima de seguimiento
solar.

\begin{verbatim}
FIGURA LAZO - SCA
\end{verbatim}

Un conjunto de SCAs con su tubo absorbedor conectado en serie constituye
un lazo. Cada lazo consta de un numuero suficiente de SCAs para
garantizar que, bajo condiciones de diseño, el fluido caloportador
alcanza la temperatura deseada a la salida del lazo, es decir, se
produce el salto térmico necesario. Si la temperatura en el SCA
sobrepasa la máxima permitida, el SCA puede desenfocar parcial o
totalmente con el fin de dejar de concentrar radiación sobre el tubo
absorbedor.

Un subcampo o sección es un conjunto de lazos conectados en paralelo, de
tal forma que se espera que el caudal que circula por cada uno de sus
lazos sea el mismo. El subcampo cuenta con válvulas de regulación de
caudal a su entrada, por lo que constituye la unidad mínima de control
de caudal en el campo solar. En algunas ocasiones cada lazo tiene
capacidad de regulación de su caudal de forma constante. En ese caso se
podría decir que cada lazo actúa como un subcampo con un único lazo,
pero esto no es lo habitual.

Finalmente, el campo solar está formado por un conjunto de subcampos. El
fluido caloportador frío entra en el campo solar y se distribuye por
cada uno de los subcampos, donde se vuelve a distribuir equitativamente
entre los lazos. En los lazos, el HTF se calienta y retorna a una
tubería que lo conduce a la salida del subcampo, donde finalmente el HTF
procendente de todos los subcampos se mezcla y se trasforta, a lo largo
de una tubería denominada colector caliente, hasta el punto de consumo.

\begin{verbatim}
DIBJO ESQUEMÁTICO CAMPO SOLAR - SUBCAMPOS Y LAZOS
\end{verbatim}

\hypertarget{clases-modelbarbero4thorder-modelbarbero1storder-y-modelbarberosimplified}{%
\paragraph{Clases ModelBarbero4thOrder, ModelBarbero1stOrder y
ModelBarberoSimplified}\label{clases-modelbarbero4thorder-modelbarbero1storder-y-modelbarberosimplified}}

Se emplean clases derivadas de la Clase Model para implementar los
diferentes modelos empleados para calcular el rendimiento y para
simular, por tanto, el funcionamiento de cada HCE. Es en cada una de
estas clases donde se desarrolla el algoritmo que, a partir de los
parámetros físicos que definen al HCE, las variables que definen el
estado del HTF que circula por él y las condiciones de operación,
resuelve las ecuaciones definidas en el modelo y nos permite conocer las
condiciones del HTF a la salida de cada HCE.

\emph{Clase ModelBarbero4thOrder}

La instancia de esta clase recibe como valores de entrada un referencia
a una instancia de un HCE del cual va a calcular su rendimiento, una
referencia a la instancia del HTF que se está empleando y valores de
condiciones meterológicas de radiación, temperatura y velocidad del
viento. El HCE debe estar inicializado previamente con los valores de
caudal másico, temperatura y presión de entrada y el flujo de calor
absorbido \(q_{abs}\).

El procedimiento de cálculo implementado en el método \emph{calc\_pr()}
es el siguiente (los parámetros que se obtienen mediante métodos propios
de las instancias del HCE y del HTF se explican en los apartados
correspondiente más adelante): * Estimación de la temperatura de pared
exterior del tubo absorbedor \(T_{ro}\) según la ec\(\eqref{eq:tro}\) a
partir del coeficiente de transmisión de calor al interior \(U_{rec}\) y
del flujo de radiación abosorbido por el tubo absorbedor,
\(\dot q''_{abs}\), a partir de la instancia del HCE. Para el primer HCE
del lazo se asume un rendimiento inicial \(\eta=1\) pero para los
siguientes se emplea el rendimiento del HCE anterior, con lo cual se
acelera un poco el proceso de convergencia por partirse de un valor
previsiblemente más próximo.

\begin{equation}
    T_{ro} = T_f + \eta \cdot \frac{\dot q''_{abs}}{U_{rec}}
    \label{eq:tro}
\end{equation}

\begin{itemize}
\tightlist
\item
  Cálculo del flujo de pérdidas \(\dot q''_{perd}\) mediante la
  ec.\(\eqref{eq:qperd}\) incrementado con las pérdidas a través de los
  soportes que sujetan el tubo absorbedor, \(\dot q_{perd,soportes}\).
  Las pérdidas en los soportes se modelan mediante la
  ec.\(\eqref{eq:qperdidassoportes}\) que se explica en el apartado
  correspondiente a la clase HCE.
\item
  Cálculo de los parámetros de funcionamiento \(\dot q''_{crit}\),
  \(U_{crit}\) y \(NTU\) para el HCE con la temperatura de pared
  calculada previamente según las ecuaciones \(\eqref{eq:qcrit}\) y
  \(\eqref{eq:ucrit}\) respectivamente.
\item
  Cálculo de los coeficientes \(f_1\), \(f_2\), \(f_3\) y \(f_4\)
  mediante las ecuaciones \(\eqref{eq:f1}\) a \(\eqref{eq:f4}\) y
  cálculo de \(f_0\) mediante la ec.\(\eqref{eq:f0}\).
\item
  Se resuelve la ec.\(\eqref{eq:rendimiento0}\) con de forma iterativa
  mediante Newton-Raphson para calcular \(\eta_0\). Como valor inicial
  se calcula \(\eta_0\) a partir de la
  ec.\(\eqref{eq:rendimiento0aproximado}\) del Modelo de \(1_{er}\)
  Orden.
\item
  Con el valor de \(\eta_0\) obtenido se calculan los valores de \(Z\),
  \(g'(Z)\), \(g''(Z)\) y \(g'''(Z)\) dados por las ecuaciones
  \(\eqref{eq:zeta}\) a \(\eqref{eq:g3primadezeta}\).
\item
  Finalmente, se calcula el rendimiento \(\eta(x^*)\) según la
  ec.\(\eqref{eq:modelocompleto}\), la temperatura de pared exterior
  \(T_{ro}\) y se comparan con los valores iniciales. Si las diferencias
  son superiores a cierto margen configurable se vuelve a realizar otra
  iteración hasta conseguir la convergencia, pero previamente a cada
  iteración se recalculan todos los pasos anteriores empleando la
  temperatura de pared del tubo absorbedor calculada con el nuevo
  rendmiento.
\end{itemize}

Para el primer HCE del lazo se emplea como temperatura del fluido
\(T_f\) la temperatura del HTF a la entrada del HCE. Para los siguientes
HCEs del lazo se incrementa la temperatura de entrada con la mitad del
salto de temperatura que experimentó el HCE anterior.

Una vez finalizado el proceso iterativo, la instancia del HCE actualiza
sus valores de rendimiento, temperatura y presión de salida del HTF,
quedando totalmente definido su punto de funcionamiento. Las condiciones
de temperatura y presión a la salida del HCE serán las de entrada del
HCE siguiente.

Al calcular el rendimiento integral para todo la longitud del HCE
estamos haciendo coincidir el tamaño de la malla de integración con la
longitud física real del HCE. Se ha comprobado que la reducción de la
malla no aumenta de forma apreciable la precisión de los cálculos y en
cambio sí supone un coste computacional importante. Por el contrario,
una forma de acelerar el proceso de simulación consiste en considerar
artificialmente que la longitud del HCE es mayor que la real. Se trata
de aumentar el tamaño de la malla de integración para reducir el número
de cálculos. En este trabajo se seguirá, al igual que en
\cite{1022085/7TD8VTGL}, el criterio de no superar un tramo de HCE
superior a 100 m propuesto en \cite{1022085/CYH3NJEG}.

\emph{Clases ModelBarbero1stOrder} y ModelBarberoSimplified

El cálculo del rendimento que realizan estas dos clases es idéntico al
del Modelo de 4º Orden hasta el momento de llegar al proceso iterativo,
punto en el cual se calcula el rendimiento térmico directamente mediante
las ecuaciones \(\eqref{eq:primerorden}\) y
\(\eqref{eq:modelosimplificado}\) respectivamente.

\hypertarget{heat-collector-element-hce}{%
\paragraph{Heat Collector Element,
HCE}\label{heat-collector-element-hce}}

Físicamente, un HCE es un tubo de acero con una envolvente de vidrio de
tal forma que entre el tubo de acero y la envolvente queda un espacio en
el que se ha practicado el vacío. Por el interior del tubo circula el
fluido caloportador (HTF, Heat Transfer Fluid) que aumenta su energía
térmica al recibir el calor procedente de las paredes interiores del
tubo. El tubo recibe durante el proceso de fabricación un recubrimiento
selectivo que mejora sus propidades físicas para absorber la raciación
solar. De cara a modelar el funcionamiento del HCE como elemento
responsable de calentar el HTF de forma compatible con el Modelo físico
desarrolladose se define la clase HCE que consta de los siguientes
atributos:

\begin{itemize}
\item
  tin: Temperatura de entrada del HTF \((K)\)
\item
  pin: Presión de entrada del HTF \((Pa)\)
\item
  massflow: Caudal másico del HTF \((Kg/s)\)
\item
  qabs: Potencia calorífica absorbida \((W/m^2)\)
\item
  qlost: Potencia calorífica perdida \((W/m^2)\)
\item
  qlost\_brackets = Potencia calorífica perdida a través de los soportes
  del HCE \((W/m^2)\)
\item
  pr: Rendimiento global del HCE ( )
\item
  pr\_opt: Rendimiento óptico del conjunto HCE + SCA ( )
\item
  tou: Temperatura de salida del HTF \((K)\)
\item
  pout: Presión de salida del HTF \((Pa)\)

  {[}FIGURA ESQUEMA CLASE HCE{]}
\end{itemize}

Con estos parámetros el comportamiento del HCE queda totalmente
caracterizado en el sistema desde el punto de vista del proceso de
generación. Estos atributos (pueden entenderse como variables) están
relacionados entre sí según las reglas que aplique cada modelo.

Ya hemos visto, al hablar de la Clases para los modelos, cómo un objeto
(instancia) de la clase HCE puede ser procesada por otra instancia de la
clase del modelo para simular su comportamiento. Es necesario que la
instancia del HCE pase los siguientes parámetros al modelo:

\begin{itemize}
\tightlist
\item
  \(k_{rec}\), conductividad térmica de la pared del receptor. Se ha
  empleado la ec.\(\eqref{eq:krec}\) válida para el acero inoxidable
  321H:
\end{itemize}

\begin{equation}
    k_{rec} = 0,0153 \cdot (t - 273,15) + 14,77
    \label{eq:krec}
\end{equation}

\begin{itemize}
\tightlist
\item
  \(h_{int}\), coeficiente de transferencia de calor convectivo hacia el
  interior. Para el cálculo se emplea la ec.\(\eqref{eq:hint}\) donde
  \(Nu_{G}\) es el número de Nusselt obtenido mediante la correlación de
  Gnielinski dada en la ec.\(\eqref{eq:nug}\), \(D_{ri}\) es el diámetro
  interior del tubo absorbedor y \(k_f\) es la conductividad térmica a
  la temperatura del fluido:
\end{itemize}

\begin{equation}
    h_{int} = \frac{Nu_{G}\cdot k_f }{D_{ri}}
    \label{eq:hint}
\end{equation}

\begin{equation}
    Nu_{G} = \frac{ \left( \frac{C_f}{2} \right)\cdot\left( Re_{D_{ri}} - 1000 \right)\cdot Pr_f }{1 + 12,7 \cdot \left(\frac{C_f}{2} \right)^{\frac{1}{2}}\cdot \left(Pr^{\frac{2}{3}}_f -1 \right)} \cdot \left( \frac{Pr_f}{Pr_{ri}} \right)^{0,11}
    \label{eq:nug}
\end{equation}

\begin{itemize}
\item
  \(U_{rec}\), coeficiente de transmisión de calor al interior. Viene
  dado por la ec.\(\eqref{eq:urec}\) comentada previamente.
\item
  \(\dot q''_{abs}\), flujo de calor que absorbe el HCE. En este caso,
  se trata de un parámetro que almacena cada instancia de HCE y que se
  calcula según la ec.\(\eqref{eq:qabs}\) que repetimos a continuación.
\end{itemize}

\(\dot q''_{abs}= \eta_{opt}(\theta) \cdot Cg \cdot DNI \cdot \eta_{sombras} \cdot \eta_{bordes}\)
\(\eqref{eq:qabs}\)

Cada uno de los parámetros de esta ecuación se obtiene de la siguiente
manera:

\begin{itemize}
\item
  \(DNI\) es la irradiancia normal directa cuyo valor se lee para cada
  fecha de cálculo de la simulación.
\item
  \(Cg\) es el factor de concentración geométria, definido genéricamente
  para sistemas de concentración como el cociente entre el área de
  apertura del concentrador, \(A_c\) y el área de apertura del receptor,
  \(A_r\). Hemos considerado como efectiva toda el área del receptor, no
  solo aquella donde se concentra la radiación ya que supondremo que el
  flujo se reparte uniformemente por toda la superfice de tubo
  absorbedor. De esta manera, el factor de concentración geométrica para
  un colector cilindroparabólico es, según la ec.\(\eqref{eq:cg}\):
\end{itemize}

\begin{equation}
   Cg = \frac{a}{\pi \dot D_{ro}}
    \label{eq:cg}
\end{equation}

donde \(a\) es la apertura de la parábola del concentrador y \(D_{ro}\)
es el diámetro exterior del tubo absorbedor.

\begin{itemize}
\tightlist
\item
  \(\eta_{opt}(\theta)\), rendimiento óptico. Este parámtro depende del
  ángulo de incidencia \(\theta\) y se obtiene a partir del rendimiento
  óptico pico, \(\eta_{opt,peak}\) y del modificador del ángulo de
  incidencia, \(IAM\) según la ec.\(\eqref{eq:rendimientooptico}\):
\end{itemize}

\begin{equation}
   \eta_{opt}(\theta) = \eta_{opt,peak} \cdot IAM \cdot cos(\theta)
    \label{eq:rendimientooptico}
\end{equation}

Para calcular \(\eta_{opt,peak}\) empleamos la expresión dada en la
ec.\(\eqref{eq:rendimientoopticopico}\). La ecuación para \(IAM\) se
ofrece en la sección correspondiente al modelado del SCAA por ser un
valor característico de este componente.

\begin{equation}
   \eta_{opt,peak} = \alpha \cdot \tau \cdot \rho \cdot \gamma
    \label{eq:rendimientoopticopico}
\end{equation}

Los parámetros \(F_0\), \(F_1\) y \(F_2\) los ofrece cada fabricante
para su concentrador. El \(IAM\) es una propiedad del SCA y por tanto la
instancia del HCE hace una llamada al método \emph{get\_IAM} de su SCA
asociado, aquel en el que está montado, para recibir su valor.

Igualmente, en la ec.\(\eqref{eq:rendimientoopticopico}\) los parámetros
\(\rho\) (reflectividad del concentrador) y \(\gamma\) (fracción solar),
son parámetros del SCA y deben obtenerse de la instancia de SCA asociada
al HCE. \(\alpha\) es la absortividad del receptor y \(\tau\) es la
transmisividad del vidrio envolvente del tubo absorbedor. En ambos casos
se trata de parámetros configurables que se introducen con el resto de
características del HCE en el archivo de configuración de la simulación.

\begin{itemize}
\tightlist
\item
  \(\eta_{bordes}\) contabiliza las pérdidas debidas a que en una
  pequeña porción del tubo absorbedor del SCA no se produce
  concentración debido al ángulo de incidencia. Un tramo del tubo
  absorbedor, que puede implicar desde solo un tramo del primer HCE
  hasta a varios HCEs, tendrá un flujo de radiación nulo, o muy bajo. El
  tramo de tubo absorbedor que queda sin concentración (\(L_{bordes}\)
  se calcula mediante la ec.\(\eqref{eq:bordes}\) a partir de la
  distancia focal, \(f_l\) y de ángulo de incidencia \(\theta\):
\end{itemize}

\begin{equation}
   L_{bordes} = \frac {f_l}{tan(\theta)}
    \label{eq:bordes}
\end{equation}

A partir de este valor el código calcula que fracción del HCE o cuantos
HCEs quedan inutilizados y les asigna un rendimiento nulo.

*{[}REVISAR MÉTODO DE CÁLCULO{]} \(\eta_{sombras}\) es un valor que se
calcula en base a la porción del concentrador que se encuetra afectado
por sombras debido a que la distancia de separación entre lazos está
acotada. En disposiciones de lazos habituales con eje seguimiento
Norte-Sur estas sombras solo aparecen a primera y última hora del día.
Su cáculo exacto requeriria conocer totalmente la disposición de cada
lazo, pero una aproximación suficiente se puede obtener mediante la
ec.\(\eqref{eq:sombras}\):

\begin{equation}
   \eta_{sombras} =  \frac {sen(\alpha_s) \cdot D_L}{A_c}
    \label{eq:sombras}
\end{equation}

Una vez que la instancia del modelo ha concluido el cálculo, el HCE ya
puede cacular cuál será la temperatura de salida del HTF, \(t_{out}\),
que aparece implícita en la ec.\(\eqref{eq:delta_h}\):

\begin{equation}
   \Delta H =  {\dot m} \cdot \int_{t_{in}}^{t_{out}}Cp(t)dt
    \label{eq:delta_h}
\end{equation}

donde \(\Delta H\) es el incremento de entalpía del H, pues hemos
considerado que se trata de un fluido incompresible y también se ha
despreciado la participación de energía cinética. Previamente se debe
calcular \(\Delta H\) según la ec.\(\eqref{eq:delta_hvalor}\):

\begin{equation}
   \Delta H =  \dot q''_{abs} \cdot \eta \cdot \pi \cdot d_{ro} \cdot L \cdot \gamma_L \cdot \gamma_g
    \label{eq:delta_hvalor}
\end{equation}

La ec.\(\eqref{eq:delta_h}\) puede resolverse por métodos numéricos si
el calor específico \(Cp\) del fluido se ha obtenido a partir de un
polinomio. En el caso de que se disponga de una función que proporcione
la temperatura del fluido en función de la entalpía \(T(H)\), como
ocurre si se usa \(CoolProp\), se puede calcular su valor directamente
como \(T_{out} = T(H_{out})\).

Se introducen el \emph{factor de longitud efectiva}, \(\gamma_L\) y
\emph{factor de interceptación geométrico}, \(\gamma_g\) para tener en
cuenta la reduccion de la longitud \emph{activa} del HCE debido a los
fuelles en los extremos del HCE y al sombreado del escudo térmico en las
uniones de HCEs. Un valor típico para ambos factores está comprendido
entre 0,96 y 0,97 \cite{1022085/XAQ9AM5E}. En el caso de que el calor
absorbido sea nulo, la temperatura de salida será inferior a la de
entrada y el valor \(\Delta H < 0\). En este caso, no existe reducción
de la longitud efectiva del absorbedor y la energía perdida se calcula
según la ec.\(\eqref{eq:delta_hvalornegativo}\) pues a lo largo de toda
la superficie del HCE se experimentan pérdidas energéticas:

\begin{equation}
   \Delta H =  \dot q''_{perd} \cdot \eta \cdot \pi \cdot d_{ro} \cdot L
    \label{eq:delta_hvalornegativo}
\end{equation}

\begin{itemize}
\tightlist
\item
  \(\varepsilon_{ext}\), emisividad equivalente de la superficie
  exterior del receptor. La emisividad equivalente empleada por el
  modelo depende de la temperatura de pared exterior del tubo y se
  emplea la ec.\(\eqref{eq:eext}\) para calcularla:
\end{itemize}

\begin{equation}
   \varepsilon =  A_0 + A_1 \cdot  (t_{ro} - 273.15)
    \label{eq:eext}
\end{equation}

Se corrige ligeramente su valor en función de la velocidad del viento,
incrementando su valor un 1\% con un viento de 4 m/s y un 2\% para
viento de 4 m/s. Los coeficientes \(A_0\) y \(A_1\) son los que se
ofrecen en \cite{1022085/7TD8VTGL}.

\begin{itemize}
\item
  \(h_{ext}\), coeficiente de transferencia de calor convectivo
  equivalente al exterior. Su valor puede considerarse nulo para el caso
  de un HCE con vacío en su espacio anular. En \cite{1022085/7TD8VTGL}
  se ofrecen las ecuaciones para diferentes combinaciones de
  recubrimiento, Black-Chrome o Cermet y conservación o no del vacío,
  obtenidas mediante simulación CFD (Computational Fluid Dynamics) por
  su autor para un modelo unidimensional del HCE.
\item
  \(\dot q_{perd,soportes}\), pérdidas a través de los soportes que
  sujetan en HCE. Se hace uso de la ec.\(\eqref{eq:qperdidassoportes}\)
  propuesta en \cite{1022085/CYH3NJEG}:
\end{itemize}

\begin{equation}
   \dot q_{perd,soportes} =  n \cdot \frac{\sqrt{P_b \cdot k_b \cdot A_{cs,b} \cdot \bar h_b} \cdot (T_{base}T{ext})}{L}
    \label{eq:qperdidassoportes}
\end{equation}

donde \(P_b\) es el perímetro de la sección del soporte, \(A_{cs,b}\) es
la sección transversal de la unión entre el brazo y el tubo absorbedor,
\(K_b\) es la conductividad térmica del acero empleado en el brazo,
\(\bar h_b\) es el coeficiente de transmisión de calor por convección
medio hacia el exterior, \(T_{base}\) es la temperatura de la zona de
conexión entre los brazos y el tubo absorbedor, \(L\) es la longitud del
colector y \(n\) es el número de soportes por colector.

La Clase HCE también nos proporciona algunos métodos necesarios para el
trabajo de procesamiento de la información, asignación y recuperación de
valores de los atributos.

Otro aspecto importante es que cada instancia de la clase HCE tiene un
atributo de tipo `diccionario', en el que a modo de lista de pares
clave-valor va a recibir aquellos parámetros que posteriormente serán
empleados por el Modelo. Los diferentes autores que han elaborado
modelos para los HCE no siempren utilizan los mismos parámetros ni
idénticos identificadores. Al emplear un diccionario se facilita la
tarea de implementación de nuevos Modelos, sin que sea necesario cambiar
los atributos de la clase en cada ocasión. Entre los parámetros que se
pasan al HCE durante la creación de su instancia están su absortividad
solar \(\alpha_{solar}\), la transmisividad del vidrio \(\tau\) y su
reflectividad \(\rho\).

En el caso de la planta simulada se ha utilizao un HCE fabricado por
Solel cuyos parámetros se guardan en una libería en formato JSON. Parte
de los datos de cada HCE de la librería se han extraído de los archivos
de configuración de SAM (System Advisor Model), el software de
referencia para la simulación de plantas de energías renovables. No
obstante, el modelo no hace uso de todos ellos y, en cambio, se precisa
de algún dato más para realizar la simulación. Estos parámetros son
almacenados en el diccionario \emph{parameters}. El programa
desarrollado permitiría, en principio, modelar cada HCE con unos
parámetros diferentes, es decir, que cada HCE se comportase de forma
diferente al resto. Esta funcionalidad puede ser interesante para el
estudio de comportamiento del campo solar cuando se dispone de
estadísticas adecuadas sobre cómo evoluciona en el tiempo y se
distribuye en el campo solar cada parámetro, lo cual hace que no todos
los lazos se comporten de igual manera. El inconveniente es que el
tiempo de cálculo aumenta notablemente al tener que simularse cada lazo
independientemente.

Finalmente, un HCE es un elemento que ocupa una determinada posición
dentro del SCA (Solar Collector Assembly). Más adelante se verá que,
para determinadas simulaciones, el orden que ocupa dentro de la fila de
HCEs y el propio SCA al que pertenece, pueden ser datos necesarios a la
hora de estudiar su comportamiento. Por este motivo cada HCE mantiene
una referencia al SCA al que pertenece y guarda información sobre su
posición relativa dentro de él.

\begin{itemize}
\tightlist
\item
  sca: Una referencia al objeto que representa el SCA en el cual el HCE
  está montado.
\item
  hce\_order: Un número entero, indicando la posición relativa del HCE
  dentro del SCA.
\end{itemize}

\hypertarget{solar-collector-assambly-sca}{%
\paragraph{Solar Collector Assambly,
SCA}\label{solar-collector-assambly-sca}}

Un SCA es una estructura compuesta por una serie de reflectores que
cocentran la radiación solar sobre los HCE. Desde el punto de vista
operativo, un SCA cuenta con capacidad de movimiento independiente
respecto al resto de SCAs de la planta, por lo que es la unidad mínima
de control de enfoque o desenfoque de la radiación solar en el campo
solar. La clase SCA nos permite modelar cada SCA teniendo en cuenta las
propiedades de los reflectores (reflectividad, suciedad de los espejos,
precisión del movimiento de seguimiento solar, etc.)

En plantas de colectores cilindroparabólicos lo más frecuente es que el
sistema de seguimiento tenga su eje de rotación alineado en la dirección
Norte-Sur con el fin de hacer un seguimiento Este-Oeste de la
trayectoria solar a lo largo del día. No obstante, una configuración con
eje Este-Oeste también puede ser interesante en algunos casos y el
modelo también permite esta configuración.

Dentro del campo solar, cada HCE debe pertenecer a un SCA. El primer HCE
del SCA recibe el fluido caloportadador procedente de otro SCA o de las
tuberías colectoras de HTF frio. El último HCE del SCA entrega el HTF
más caliente al siguiente SCA o a las tuberías colectoras de HTF
caliente a la salida del lazo.

El SCA cuenta con un método para el cálculo del modificador por ángulo
de incidencia. Hemos considerado que el SCA mantendrá en todo momento un
ángulo de seguimiento \(\beta\) óptimo con el fin de minimizar el ángulo
de incidencia. En este caso, el \(IAM\) se calcula según la expresión
dada por la ec.\(\eqref{eq:iam}\)

\begin{equation}
   IAM = F_0 + F_1 \cdot \frac{\theta}{cos(\theta)} + F_2 \cdot \frac{\theta^2}{cos(\theta)},\, \forall   
 \theta \in (0^o, 80^o)
    \label{eq:iam}
\end{equation}

Algunos fabricantes incluyen un factor \(cos(\theta)\) en la expresión
del \(IAM\), por lo que no deberá incluirse entonces en la ecuación del
rendimiento total del HCE. En nuestro caso, para el UVAC 3 de Solel
empleado en la simulación, no es así.

Otro valor que debe ofrecernos la clase que modela al SCA es la
\emph{fraccción solar} o \emph{factor de interceptación}, que permite
estimar la tasa entre la radiación solar que alcanza al reflector y la
que posteriormente indice realmente sobre el tubo absorbedor. Su valor
se optienen según la ec.\(\eqref{eq:fraccionsolar}\) como producto de
una serie de factores:

\begin{equation}
   \gamma = \eta_{geometrico} \cdot \eta_{seguidor} \cdot \eta_{suciedad} \cdot \eta_{disponibilidad}
    \label{eq:fraccionsolar}
\end{equation}

El factor geométrico \(\eta_{geometrico}\) depende de las inperfecciones
geométricas de conjunto reflector-absorbedor como pequeñas desviaciones
en la curvatura de los espejos o la deformación de la estructura. El
factor de precisión del seguidor \(\eta_{seguidor}\) permite considerar
los errores de seguimiento del mecanismo de movimiento del reflector. El
factor de suciedad \(\eta_{suciedad}\) se refiere a las pérdidas de
reflectividad debidas a acumulación de polvo en los espejos. En
realidad, si no se ha considerado un equivalente para el polvo acumulado
en la superficie del vidrio del HCE, este factor debería computarse dos
veces, pues la merma de radiación se produce tanto en el espejo, como
radiación solar no reflejada, como en la envolvente de vidrio, al
disminuir su transmisividad. Finalmente, el factor de disponibilidad
\(\eta_{disponibilidad}\) considera las pérdidas que ocasionalmente se
puedan producir por averias del sistema de concentración.

El SCA, como sistema responsable del seguimiento solar, también cuenta
con un método para ofrecernos información sobre el ángulo de incidencia
\(\beta\) en el plano de apertura del reflector. Las expresiones
generales pueden encontrarse en \cite{1022085/95AM6AQN}. En nuestro caso
hemos recurrido a las librería \emph{pvlib-python}
\cite{1022085/GUC54R5I} desarrollada en Sandia National Laboratories
para obtener los valores del ángulo de incidencia y la posición solar
para cada fecha del año y según las coordenadas geográficas del lugar
donde se realiza la simulación.

\hypertarget{lazo}{%
\paragraph{Lazo}\label{lazo}}

Un Loop o lazo es un cojunto de SCAs conectados en serie de tal forma
que el HTF que entra frío al lazo experimenta un salto térmico cuando
transita por él. El sistema de control ajusta el estado de enfoque o
desenfoque de cada SCA en el lazo con el fin de conseguir que la
temperatura de salida sea la de consigna. Por motivos de económicos, el
caudal de HTF no suele ser regulable a nivel de lazo, pues obligaría a
instalar una válvula de control en cada uno de ellos y por tanto todos
los lazos de un mismo subcampo suelen tener un caudal muy parecido. Se
ha desarrollado también una Clase Subfield para dar cuenta del conjunto
de lazos que pertenecen a un mismo subcampo y, por tanto, pueden variar
su caudal de forma independiente de los lazos de otro subcampo. En un
campo solar suele haber haber varios subcampos que pueden regular su
caudal independientemente. Los casos más extremos serían el de un campo
solar con un único subcampo, en el que todos los lazos pertenecen al
mismo subcampo y en el otro extremo, un campo solar con tantos subcampos
como lazos tiene, de tal forma que cada lazo es el único en su subcampo
y por tanto cada lazo puede regular su caudal independientemente.

Cada lazo del campo solar es modelado mediante una instancia de la clase
Loop. Cada instancia mantiene referencias al subcampo al que pertenece y
a los SCA que contiene. Atributos importantes de estas instancias serán
el caudal de HTF en el lazo, las temperaturas y presiones de entrada y
salidad del HTF y el rendimiento completo del lazo.

El código permite trabajar de dos formas, bien calculando y ajustando el
caudal requerido para conseguir una temperatura de HTF determinada o
bien fijando el caudal de HTF y calculando qué temperatura de salida
tendría el HTF. En caso de que esta tempertura de salida supere el
máximo permitido (un valor configurable) el código considera que se
producirá un desenfoque en el SCA en que se alcance esta temperatura y
el HTF dejará de calentarse. En este caso se suele decir que se produce
un vertido de energía o desaprovechamiento de la radiación existente. El
código permite contabilizar esta energía desaprovechada por cada lazo
durante estas situaciones.

La Clase PrototypeLoop hereda de la Clase Loop sus principales métodos y
atributos pero supone una pequeña variación de una instancia o lazo
definido por Loop ya que se trata de un lazo ``típico'' o ``promedio'',
que presenta una configuración constructiva idéntica a la del resto de
lazos de la planta pero no pertenece a ningún subcampo solar. Este lazo
especial o prototipo se empleará cuando queramos realizar un estudio
paramétrico del comportamiento del lazo, no de la planta, y también para
realizar una simulación mucho más rápida cuando asumamos la hipótesis de
que todos los lazos de la planta se comportan de igual manera. Esta
aproximación es la que se hace en aplicaciones como SAM, donde no se
simulan todos los lazos para modelar el campo solar, sino que se simula
solo un lazo y el resultado se obtiene multiplicando el caudal de salida
de este lazo por el número de lazos que forman la planta.

\hypertarget{subfield}{%
\paragraph{Subfield}\label{subfield}}

Un subcampo es un conjunto de lazos en los que se considera que el HTF
se repartirá equitativamente. Cada subcampo dispone de una válvula de
control de caudal a su entrada y representa el mayor grado de control de
caudal de HTF que se puede alcanzar en el campo solar. Un campo solar
suele contar con varios subcampos y cada uno de ellos, a su vez, cuenta
con varios lazos.

La Clase Subfield mantiene referencias a lo lazos que lo constituyen, es
deir, a cada una de las instancias que representan un lazo. También
mantiene referencia al campo solar al que pertenece y cuenta con métodos
para calcular cual será la tempertura de salida del HTF del subcampo
solar una vez que el caudal de salida de cada lazo se haya mezclado con
el del resto de lazos. Nótese que aunque se supone que todos los lazos
del subcampo tienen el mismo caudal másico la temperatura de salida de
cada uno de ellos puede ser diferente y, por tanto, la energía aportada
por cada lazo también lo será.

\hypertarget{solarfield}{%
\paragraph{Solarfield}\label{solarfield}}

El campo solar alberga el conjunto de subcampos y, por tanto, todos los
lazos de la instalación. Se trata del objeto que en última instancia
queremos modelar con el fin de conocer cómo sera el comportamiento de la
planta solar y su rendimento anual. A la hora de definir el campo solar
para el modelo es necesario conocer cuántos subcampos contiene, cuántos
lazos hay en cada subcampo, qué configuración tiene cada lazo (número de
SCAs en cada lazo y número de HCEs en cada SCA). También es importante
conocer la distancia entre lazos con el fin de estimar el sombreado que
se produce a primera y última hora del día.

El Clase Solarfield también recibe una serie de valores nominales para
los que se ha realizado un diseño óptimo del sistema, como las
temperaturas y presiones de entrada y salida del HTF, las temperaturas
máximas y mínmas tolerables por razones de seguridad, el caudal nominal
(normalmente se suele dar este caudal por lazo y el caudal del campo
solar será la suma de todos ellos), el caudal mínimo (existe cierta
limitación tanto por la velocidad mínima de las bombas como por otras
cuestiones operativas que desaconsejan que el HTF circule por debajo de
este límite).

Cuando se crea una instancia de la Clase Solarfield, ésta recibe los
datos de configuración que se ha pasado para la simulación y lanza el
proceso de construcción que genera, en base a esa configuración, los
diferentes subcampos, lazos, SCAs y HCEs que forma en campo solar.

\hypertarget{clase-fluid-y-sus-clases-hijas-fluidcoolprop-y-fluidtabular}{%
\paragraph{Clase Fluid y sus clases hijas, FluidCoolProp y
FluidTabular}\label{clase-fluid-y-sus-clases-hijas-fluidcoolprop-y-fluidtabular}}

Para modelar el HTF (Heat Transfer Fluid) se ha creado una Clase Fluid.
Las propiedades del HTF pueden obtenerse mediante funciones polinomicas
con coeficientes constantes calculados experimentalmente o desde
librerias preexistentes como CoolProp. Por ese motivo, según se opte por
un método u otro, se han creado las subclases FluidCoolProp y
FluidTabular. No obstante, se ha comprobado que el número de tipos de
HTF que existen en la librería FluidCoolProp no es muy grande,
limitándose a Therminol VP-1, Syltherm 800. No se encuentra en esta
librería el aceite Dowtherm A, que es el que se emplea en la planta
cuyos datos se han empleado para el desarrollo de esta herramienta. Pero
el mayor inconveniente reside en que CoolProp devuelve valores solo
dentro del rango de temperaturas de uso válidas según el fabricante.
Este rango es demasiado estricto y se producen problemas debido a la
devolución de valores no numéricos, especialmente cuando se está
calculando la temperatura teórica de salida del HTF cuando hay
sobrecalentamiento. Por este motivo se ha hecho uso mayoritariamente de
la clase FluidTabular, con funciones polinómicas con coeficientes
calculados a partir de los datos ofrecidos por los fabricantes.

La clase Fluid sus clases hijas ofrecen métodos para calcular la
densidad, viscosidad cinemática, número de Reynolds, calor específico,
conductividad térmica y entalpia en función de la temperatura y la
presión. También ofrecen un método para calcular la temperatura del
fludo en función de la entalpía y la presión, considerando entalpía cero
para líquido saturado según ASHRAE a una tempratura de 285.856 K.

\hypertarget{clases-weather-fielddata-y-tabledata}{%
\paragraph{Clases Weather, FieldData y
TableData}\label{clases-weather-fielddata-y-tabledata}}

Estas clases impolemetan métodos adecuados para la adquisición de datos
desde diferentes tipos de ficheros, en concreto:

\begin{itemize}
\tightlist
\item
  Clase Weather para ficheros .tmy con datos meteorológicos (Weather
  Files). En estos ficheros solo hay datos meteorológicos como radiación
  normal incidente (DNI), temperatura de bulbo seco, y datos de
  geográificos del emplazamiento (Site), como latitud, longitud y
  altitud. A partir de estos datos se pueden realizar simulaciones para
  ver cuál sería el comportamiento de la planta con estas condiciones.
\item
  Clase FieldData para ficheros .csv con datos recogidos de alguna
  planta (Field Data Files). Estos ficheros contienen dastos
  meteorológicos recogidos por la estaciones de planta y también datos
  de instrumentación de planta, en concreto, temperaturas y presiones de
  entrada y salida al campo solar y los diferentes subcampos y también
  caudales másicos en los subcampos. Los encabezados de cada columna
  probablemente serán identificadores o tags propios de cada planta, por
  lo que es necesario indicar al programa a qué dato corresponde cada
  tag. Esto se puede hacer en el fichero de configuracion de la
  simulación. Con estos datos se puede simular el comportamiento del
  campo solar para caudal teórico requerido pero también comprobar cuál
  sería el rendimiento del campo solar operando con el caudal real de
  planta. Los datos obtenidos podrán después compararse con los reales
  de funcionamiento de planta. A este tipo de simulaciones las
  denominaremos `benchmark'.
\item
  Clase TableData para ficheros .csv empleados en simulaciones
  distintas, por ejemplo para el estudio del rendimiento de un lazo en
  función de diferentes valores de \(q_{abs}\).
\end{itemize}

\hypertarget{clases-site}{%
\paragraph{Clases Site}\label{clases-site}}

La Clase Site (Emplazamiento), contiene la información relativa al lugar
donde está ubicada la planta. Los datos de latitud, longitud y altitud
son importantes a la hora de calcular la trayectoria solar para cada
fecha. Nos ofrece un método para calcular la posición del sol en cada
fecha del año en base a los parámetros que almacenan las coordenadas
geográficas.

\hypertarget{algoritmo-de-simulacion}{%
\subsubsection{Algoritmo de simulacion}\label{algoritmo-de-simulacion}}

En este apartado se describe como puede desarrollarse un código a partir
de las Clases implementadas y comentadas anteriormente con el fin de
realizar diferentes tipos de simulaciones. Para ello, se continua con la
filosofía de POO y se crea desarrolla la clase SolarFieldSimulation.

El objetivo que nos proponemos es simular el comportamiento de un campo
solar bajo unas determinadas condiciones. Puesto que estas condiciones
varían a lo largo del día, se emplearán ficheros de datos en formato
tabular que cuentan con una columna índice para la fecha y hora
indicadas. Con el fin de poder reaprovechar el trabajo realizado durante
el trabajo de configuración de la simulación, se emplea un archivo en
formato JSON que recoge todos los parámetros necesarios. En resumen, la
instancia de SolarFieldSimulation recibe realiza los siguientes pasos:

\begin{itemize}
\tightlist
\item
  Lee el archivo de configuración de la simulación y almacena los
  parámetros necesarios.
\item
  Crea una instancia de la Clase Site con información sobre la ubicación
  de la planta.
\item
  Crea una instancia para el almacenamiento de los datos del fichero en
  formato tipo tabla. En función de si el fichero es de tipo
  meteorológico (TMY2 o TMY3) o es un fichero en formato CSV creará una
  instancia de la clase Weather o FieldData respectivamente. Los datos
  cargados se almacenan en un DataFrame de la librería Pandas denominado
  \emph{datasource}.
\item
  Crea una instancia para el modelado del HTF a partir de la Clase
  FluidCoolProp si los datos se van a tomar desde la libería externa
  CoolProp o de la Clase FluidTabular si se le pasan los factores de los
  polinomios que permiten calcular cada parámetro del fluido.
\item
  Crea una instancia SolarField a partir de los parámetros de
  configuración de campo solar.
\item
  Crea una instancia BaseLoop a partir de los parámetros de
  configuración de lazo.
\end{itemize}

A partir de aquí, la instancia de SolarFieldSimulation ya dispone de lo
necesario para realizar la simulación del campo mediante su método
\emph{runSimulation()}.

El tipo de simulación que se realiza depende del tipo de datos de que se
disponga y de lo que se seleccione en el archivo de configuración:

\begin{itemize}
\tightlist
\item
  Simulación tipo \emph{simulation}: En este caso, el caudal del lazo se
  recalculará en un proceso de convergencia hasta conseguir que la
  temperatura de salida del lazo sea la temperatura consignada.

  \begin{itemize}
  \tightlist
  \item
    Si el tipo de datos del que se dispone no tiene datos reales de
    planta, la instancia \emph{datasource} será de tipo Weather y solo
    se cuenta con datos meteorológicos (\(DNI\), \(T_{ext}\), velocidad
    del viento y presión atmosférica), por lo que la temperatura de
    entrada a los lazos será la nominal
  \item
    Si el tipo de datos del que se dispone sí cuenta con datos reales de
    planta que permitan conocer las temperaturas de entrada a los lazos
    (como es nuestro caso), la simulación utilizará estas temperaturas a
    la hora de ajustar los caudales al salto térmico necesario.
  \end{itemize}
\item
  Simualción tipo \emph{benchmark}: En este caso se debe disponer
  obligatoriamente de datos reales de planta, pues la simulación utiliza
  las temperaturas de entrada a los lazos y los caudales reales para
  calcular cúal será la temperatura de salida. Posteriormente, en los
  archivos de salida de datos, se puede comparar la temperatura real de
  salida del lazo con la calculada y de esta forma estimar si ha habido
  desenfoque y por tanto, desaprovechamiento de la energía solar. Hay
  que tener en cuenta en los datos que disponemos podemos encontrar
  situaciones en las que el lazo alcanza su temperatura de consigna pero
  es posible que se estuviera realizando un ajuste de enfoque-desenfoque
  (ya que el caudal no es regulable a nivel de lazo). En ese caso, es
  intersante saber qué temperatura hubera alcanzado el HTF de no haberse
  producido el desenfoque y, por tanto, poder calcular la energía solar
  que no se ha aprovechado por no poder introducir mayor caudal en el
  lazo.
\end{itemize}

A la hora de realizarse cada una de estas simulaciones puede darse el
caso de que se haya configurado la opción \emph{fastmode=True}. En este
caso se considera que todos los lazos de la planta se comportan como el
lazo típico, modelado mediante la Clase BaseLoop. En caso contrario, la
simulación se realizará para cada lazo del campo, lo cual solo tiene
sentido si los lazos o sus componentes cuentan con diferentes valores en
sus parámetros.

Una vez que el método runSimulation() ha procesado todas las filas
seleccionadas del DataFrame \emph{datasource}, los datos calculados que
se han ido añadiendo al DataFrame son volcados a un archivo CSV para su
posterior análisis.

\hypertarget{simulaciuxf3n-tipo-simulation}{%
\paragraph{\texorpdfstring{Simulación tipo
\emph{simulation}}{Simulación tipo simulation}}\label{simulaciuxf3n-tipo-simulation}}

Tal y como se ha mencionado, en este caso se simula la planta de tal
forma que se calcula el caudal necesario en cada lazo con el fin de que
se alcance la temperatura de consigna o nominal. En este tipo de
simulación solo son necesarios los datos meteorológicos y los valores
nominales de temperatura y presión de entrada y salida. Los valores
exactos de la presión no son necesarios y basta con seleccionar
presiones suficientemente altas para que garanticen el estado de líquido
saturado del HTF en todo momento. Las propiedades del HTF en estado de
líquido saturado no dependen significativamente de la presión.

Los datos meteorológicos se pueden leer de un fichero en formato TMY2,
TMY3 o CSV convenientemente etiquetado. Si además se dispone de los
datos de temperatura de entrada del HTF al campo solar se emplearán
éstos en lugar del valor nominal, por lo que el salto térmico necesario
se calculará partiendo de la temperatura real de entrada.

El programa calcula el rendimiento del lazo base o de cada lazo (según
se haya seleccionado o no la opción \emph{fastmode}) ajusando
iterativamente el caudal hasta que la temperatura de salida sea la
consignada. Durante las horas nocturnas el caudal es fijo e igual al
mínimo recomendado, tal y como ocurre en plantas reales.

\hypertarget{simulaciuxf3n-tipo-benchmark}{%
\paragraph{\texorpdfstring{Simulación tipo
\emph{benchmark}}{Simulación tipo benchmark}}\label{simulaciuxf3n-tipo-benchmark}}

En este tipo de simulación el caudal viene marcado por el caudal real de
planta, por lo que para este tipo de simulación es obligatorio disponer
de los datos reales. Al emplearse un caudal fijo, la temperatura
alcanzada por el HTF en la simulación no tiene porqué coincider con la
temperatura real. Dependiendo de si el valor obtenido es mayor o menor
que el real se pueden sacar las siguientes interpretaciones: * Si la
temperatura calculada es menor que la real podemos interpretar que la
simulación ha sobredimensionado las pérdidas en la configuración de la
planta. Pueden corregirse algunos parámetros o introducir un factor
multiplicativo. * Si la temperatura calculada es mayor que la real puede
deberse a que se han infravalorado las pérdidas, lo cual se podría
corregir ajustando los valores de estas en el archivo de configuración,
pero también puede deberse a que el campo real estaba realizando
desenfoques parciales para reducir la energía captada. En ese caso, la
energía no aprovechada se puede calcular como la diferencia entre la
entalpía del HTF a la temperatura de salida simulada menos su entalpia a
la temperatura de salida real.

    \hypertarget{validaciuxf3n-y-configuraciuxf3n-del-campo-solar-real}{%
\subsection{Validación y configuración del campo solar
real}\label{validaciuxf3n-y-configuraciuxf3n-del-campo-solar-real}}

Con el fin realizar una primera validación de nuestro código se
compararán en primer lugar los valores obtenidos para una determinada
configuración de campo con los que se optienen mediante SAM,
\cite{1022085/NBJ6NM3F}, un software de reconocido prestigio muy
empleado en el sector. Las posibilidades de SAM van mucho más allá del
alcance de nuestro código, por lo que nos limitaremos a comparar los
datos relativos exclusivametne al campo solar. Durante todo este trabajo
se empleará la misma configuración de campo solar que se describe a
continuación por disponerse de los datos reales de generación de dicha
planta. Posteriormente se compararán los datos reales de generación con
los resultados de una simulación realizada con SAM y con los de nuestro
programa.

\hypertarget{configuraciuxf3n-del-campo-solar-real}{%
\subsubsection{Configuración del campo solar
real}\label{configuraciuxf3n-del-campo-solar-real}}

Las plantas Aste 1A y 1B se encuentran situadas en el término municipal
de Alcázar de San Juan, provincia de Ciudad Real. Sus coordenadas
geográficas son (39,1°N 3,16°W) y la altitud es de 651 m sobre el nivel
del mar.

La potencia eléctrica nomminal de cada una de ellas es de 50 MW. El
proyecto inicial consideraba que las plantas contarían con
almacenamiento térmico, el cual se construiría durante una segunda fase
que finalmente no se llegó a ejecutar, por lo que en la actualidad solo
existe generación durante las horas de sol. Se emplearán los datos de
Aste 1B, cuya configuración es la siguente:

El campo solar cuenta con 120 lazos distribuidos de manera irregular en
4 subcampos.

\begin{itemize}
\tightlist
\item
  Subcampo NO, 31 lazos.
\item
  Subcampo NE, 28 lazos.
\item
  Subcampo SO, 27 lazos.
\item
  Subcampo SE, 34 lazos.
\end{itemize}

Todos los lazos son idénticos, contando con 4 SCAs cada uno en una
configuración tipo \(U\). El eje de seguimiento perfectamente plano se
encuentra alineado en direccion N-S. Cada SCA cuenta con un total de 336
espejos de vídrio fabricados por Flabeg.

Los SCA, modelo SenerTrough I diseñado por Sener, se han parametrizado
de la siguiente manera:

\begin{itemize}
\tightlist
\item
  Longitud: 148,5 m.
\item
  Apertura: 5,77 m.
\item
  Distancia Focal: 2,1 m.
\item
  Coeficientes del \(IAM\): \(F_{0}=1\), \(F_{1}=0,0506\),
  \(F_{2}=-0,1763\).
\item
  Precisión del seguidor: 0,99.
\item
  Precisión de la geometría: 0,98.
\item
  Disponibilidad: 0,99.
\item
  Reflectividad: 0,935.
\item
  Limpieza: 0,98.
\end{itemize}

El HCE es el modelo UVAC 3, de Solel con los siguientes parámetros:

\begin{itemize}
\tightlist
\item
  Logintud: 4,05 m.
\item
  Factor de longitud efectiva, \(\gamma_L\): 0,96.
\item
  Factor de interceptación geométrica, \(\gamma_g\): 0,96.
\item
  Diámetro interior del tubo absorbedor, \(d_{ri}\): 0,066 m.
\item
  Diámetro exterior del tubo absorbedor, \(d_{ro}\)r: 0,070 m.
\item
  Diámetro interior de la envolvente de vídrio, \(d_{gi}\): 0,115 (m)
\item
  Diámetro exterior de la envolvente de vídrio, \(d_{go}\): 0,121 (m)
\item
  Rugosidad interior: 0,000045 m.
\item
  Factor de emisividad, \(A_1\): 0,000206.
\item
  Factor de emisividad, \(A_0\): 0,0430.
\item
  Absortibidad solar del receptor: 0,96.
\item
  Transmisividad del vídrio: 0,96.
\item
  Valor mínimo del número de Reynolds recomendado: 2400.
\item
  Distancia de separación entre brazos de soportación: 4,05 m.
\end{itemize}

La planta emplea como fluído caloportador Dowtherm-A, de Dow Chemical.
Para el modelado sería necesario conocer una serie de parámetros en
función de la temperatura y la presión, pero en una planta termosolar se
debe trabajar siempre con el fluido en estado de líquido saturado y se
ha comprobado que la variación de estos parámetros con la presión es
despreciable, por lo que todos pueden expresarse a partir de un
polinomio de mayor o menor grado en función exclusivamente de la
temperatura. También se ha obtenido una curva para calcular la
temperatura del líquido saturado en función de su entalpía.

\begin{itemize}
\tightlist
\item
  \(\rho(T)\): Densidad, \((kg/m^3)\).
\item
  \(\mu(T)\): Viscosidad dinámica, \((Pa \cdot m)\).
\item
  \(k_T(T)\): Conductividad térmica, \((W/m \cdot K)\).
\item
  \(C_p(T)\): Calor específico a presión constante, \(J/Kg\cdot K\).
\item
  \(H(T)\): Entalpía específica, \(J/Kg\).
\item
  \(T(H)\): Temperatura, \((K)\), considerando H=0 a
  \(T_{ref}= 285,86 K\)
\end{itemize}

La fórmula general para cada uno de estos parámetros es del tipo de la
ec.\(\eqref{eq:polinomiohtf}\):

\begin{equation}
   parámetro(T) = a_0 + a_1 \cdot T + a_2 \cdot T^2 + a_3 \cdot T^3 + a_4 \cdot T^4 + a_5 \cdot T^5 + a_6 \cdot T^6 +  a_7 \cdot T^7 + a_8 \cdot T^8   
    \label{eq:polinomiohtf}
\end{equation}

En las siguientes figuras pueden verse los diferentes parametros del
Dowtherm-A en función de la temperatura.

\begin{figure}
\includegraphics[scale=0.8]{images/curva_densidad.png}
\caption{Densidad en función de la temperatura.} 
\label{fig:curvadensidad}
\end{figure}

\begin{figure}
\includegraphics[scale=0.8]{images/curva_viscosidad.png}
\caption{Viscosidad dinámica en función de la temperatura.} 
\label{fig:curvaviscosidad}
\end{figure}

\begin{figure}
\includegraphics[scale=0.8]{images/curva_conductividad_termica.png}
\caption{Conductividad térmica en función de la temperatura.} 
\label{fig:curvaconductividad}
\end{figure}

\begin{figure}
\includegraphics[scale=0.8]{images/curva_entalpia.png}
\caption{Entalpía en función de la temperatura.} 
\label{fig:curvaentalpia}
\end{figure}

\begin{figure}
\includegraphics[scale=0.8]{images/curva_temperatura_entalpia.png}
\caption{Temperatura en función de la entalpía.} 
\label{fig:curvatemperatura}
\end{figure}

Para el caso del Dowtherm-A se han obtenido los coeficientes de los
polinómios que caracterizan cada parámetro a partir de
\cite{1022085/C5UIUT2V} y se han contrastado las curvas con los datos
ofrecidos en \cite{1022085/LB9BTEXK}. En el caso de la viscosidad
cinemática se ha detectado que el polinomio de 8º grado que se optiene
con los coeficientes de la primera referencia presenta una gran
desviación y crecimiento asintótico para temperaturas ligeramente
superiores a la máxima de operación del fluido. Con el fin de poder
flexibilizar el proceso de cálculo y que no se produzcan desbordamientos
se ha ajustado un nuevo polinomio tras extender los datos de la
viscosidad dinámica hasta unos 450 °C aproximadamente según la tendencia
observada en el último tramo de la curva \(\mu(T)\). De esta forma se
obtiene un nuevo polinomio, con mejor comportammiento en este rango
extendido.

El mismo procedimiento se ha empleado para obtener los polinomios
característicos del fluido Therminol VP-1, aunque en este caso los
polinomios se han ajustado a partir de una lista de valores sacada de la
librería CoolProp. Pese a que el programa de simulación permite que
durante el tiempo de ejecución se obtengan los parámetros citados a
través de esta librería, existe una limitación debido al rígido margen
de temperaturas con el que esta librería trabaja para cada fluído y esto
provoca que para valores de temperatura ligeramente superiores al rango
de operación ofrecido por fabricante no se devuelva ningún valor. Esto
resulta en problemas en tiempo de ejecución si algún lazo alcanza una
temperatura superior a los 397°C (398°C en el caso de Syltherm 800) .
Aunque superar esta temperatura no es recomendable, es algo que
puntualmente ocurre durante la operación de la planta. Además, una forma
calcular la energía desaprovechada por desenfoque sería a partir de la
temperatura que hubiera alcanzado el lazo de no haberse producido el
desenfoque y calculando posteriormente su entalpía. Esta aproximación,
que sería imposíble en la vida real debido a la degradación del HTF e
incluso al daño del propio sistema por las sobrepresiones que se
producirían, facilita el cálculo de la energía desaprovechada en cada
momento. Se ha supuesto que las curvas de los parámetros se mantienen
bien ajustadas siempre y cuando la sobretemperatura alcanzada no sea
excesiva (en las simulaciones realizadas no se ha superado más del 10\%
de la temperatura máxima de operación recomendada por el fabricante).
Por estos motivos, para este trabajo se han empleado siempre los valores
de los parámetros obtenidos a partir de los polinomios y no de CoolProp.

A continuación, en la siguiente tabla, se muestran los coeficientes para
Dowtherm A:

Coeficientes de los polinomios de ajuste para Dowtherm A

\begin{longtable}[]{@{}lllllll@{}}
\toprule
& \(C_p(T)\) & \(\mu(T)\) & \(\rho(T)\) & \(k_T(T)\) & \(H(T)\) &
\(T(H)\)\tabularnewline
\midrule
\endhead
\(a_0\) & -2,363E+03 & 1,583E+00 & 1,492E+03 & 1,856E-01 & -6,511E+05 &
2,853E+02\tabularnewline
\(a_1\) & 3,946E+01 & -2,338E-02 & -3,332E+00 & -1,600E-04 & 4,121E+03 &
6,207E-04\tabularnewline
\(a_2\) & -1,702E-01 & 1,504E-04 & 1,248E-02 & 5,913E-12 & -1,235E+01 &
-1,822E-10\tabularnewline
\(a_3\) & 3,904E-04 & -5,489E-07 & -2,968E-05 & 0 & 2,771E-02 &
-1,423E-16\tabularnewline
\(a_4\) & -4,422E-07 & 1,242E-09 & 3,444E-08 & 0 & -2,776E-05 &
3,316E-22\tabularnewline
\(a_5\) & 1,979E-10 & -1,781E-12 & -1,622E-11 & 0 & 1,106E-08 &
-1,753E-28\tabularnewline
\(a_6\) & 0 & 1,581E-15 & 0 & 0 & 0 & 0\tabularnewline
\(a_7\) & 0 & -7,941E-19 & 0 & 0 & 0 & 0\tabularnewline
\(a_8\) & 0 & 1,728E-22 & 0 & 0 & 0 & 0\tabularnewline
\bottomrule
\end{longtable}

Para el caso del Therminol VP-1 los coeficientes son los siguientes:

Coeficientes de los polinomios de ajuste para Therminol VP-1

\begin{longtable}[]{@{}lllllll@{}}
\toprule
& \(C_p(T)\) & \(\mu(T)\) & \(\rho(T)\) & \(k_T(T)\) & \(H(T)\) &
\(T(H)\)\tabularnewline
\midrule
\endhead
\(a_0\) & 2,881E+02 & 1,487E+00 & 1,403E+03 & 1,486E-01 & -2,923E+05 &
2,924E+02\tabularnewline
\(a_1\) & 5,875E+00 & -2,186E-02 & -1,613E+00 & 9,755E-06 & 3,910E+02 &
6,424E-04\tabularnewline
\(a_2\) & -6,857E-03 & 1,400E-04 & 2,138E-03 & -1,780E-07 & 2,076E+00 &
-3,396E-10\tabularnewline
\(a_3\) & 4,844E-06 & -5,092E-07 & -1,931E-06 & 3,524E-12 & 1,811E-03 &
2,587E-16\tabularnewline
\(a_4\) & 6,960E-20 & 1,148E-09 & -9,610E-21 & -7,572E-25 & -1,089E-05 &
-1,066E-22\tabularnewline
\(a_5\) & -2,780E-23 & -1,642E-12 & 3,864E-24 & 2,948E-28 & 2,274E-08 &
0,000E+00\tabularnewline
\(a_6\) & 0,000E+00 & 1,454E-15 & 0,000E+00 & 0,000E+00 & -2,667E-11 &
0,000E+00\tabularnewline
\(a_7\) & 0,000E+00 & -7,286E-19 & 0,000E+00 & 0,000E+00 & 1,788E-14 &
0,000E+00\tabularnewline
\(a_8\) & 0,000E+00 & 1,583E-22 & 0,000E+00 & 0,000E+00 & -5,284E-18 &
0,000E+00\tabularnewline
\bottomrule
\end{longtable}

Con el fin de facilitar la creación del archivo de configuración que
emplea el código para realizar la simulación se ha desarrollado una
sencilla interfaz que sirve de guía durante el proceso. Se muestran a
continuación las capturas de pantalla de dicha interfaz durante la
configuración de la simulación:

\begin{figure}
\includegraphics[scale=0.8]{images/interface01.png}
\caption{Configuración de la simulación. Selección del tipo de simulación, fichero de datos y lugar de emplazamiento.} 
\label{fig:interface01}
\end{figure}

\begin{figure}
\includegraphics[scale=0.8]{images/interface02.png}
\caption{Configuración de la simulación. Configuración del campo solar, número de subcampos, lazos, configuración de los lazos, valores nominales y asistente para relacionar los identificadores de las columnas del archivo de origen de datos con los que maneja el programa.} 
\label{fig:interface02}
\end{figure}

\begin{figure}
\includegraphics[scale=0.8]{images/interface03.png}
\caption{Configuración de la simulación. Selección y configuración del fluido caloportador.} 
\label{fig:interface03}
\end{figure}

\begin{figure}
\includegraphics[scale=0.8]{images/interface04.png}
\caption{Configuración de la simulación. Selección del modelo de SCA y configuración} 
\label{fig:interface04}
\end{figure}

\begin{figure}
\includegraphics[scale=0.8]{images/interface05.png}
\caption{Configuración de la simulación. Selección del modelo de HCE y configuración} 
\label{fig:interface05}
\end{figure}

El archivo de configuración completo resultant en formato JSON puede
encontrarse en el ANEXO.

Por otra parte, la versión de SAM utilizada ha sido la \emph{2020.2.29,
64 bit, updated to revision 1}. Se muestran también en las siguientes
figuras las capturas de pantalla de las etapas más relevantes del
proceso de configuración.

\begin{figure}
\includegraphics[scale=0.8]{images/captura_sam01.png}
\caption{Configuración SAM. Campo solar} 
\label{fig:captura01}
\end{figure}

\begin{figure}
\includegraphics[scale=0.8]{images/captura_sam02.png}
\caption{Configuración SAM. Configuración del SCA SenerTrough I a partir del modelo EuroTrough ET150} 
\label{fig:captura02}
\end{figure}

\begin{figure}
\includegraphics[scale=0.8]{images/captura_sam03.png}
\caption{Configuración SAM. Configuración del HCE UVAC 3 con vacío}
\label{fig:captura03}
\end{figure}

\begin{figure}
\includegraphics[scale=0.8]{images/captura_sam04.png}
\caption{Configuración SAM. Configuración del bloque de potencia} 
\label{fig:captura04}
\end{figure}

\hypertarget{resultados-de-la-validaciuxf3n}{%
\subsubsection{Resultados de la
validación}\label{resultados-de-la-validaciuxf3n}}

Una vez que tenemos configurado SAM y nuestro programa de la misma
manera ejecutamos SAM sobre un archivo con los datos meteorológicos de
un año y en valores horarios, en concreto se trata de los datos del año
2009 que se emplearon para el estudio y anteproyecto de construcción de
las plantas.

Los datos generados por SAM son exportados a un archivo en formato
\emph{.csv} y parte de ellos los emplearemos como datos de entrada en
nuestro programa: * Temperatura de entrada al campo solar: Usaremos este
datos para que en podamos comparar entre ambas simulaciones los saltos
de temperatura respecto a una misma temperatura. Nuestro código
desconoce qué energía se extraerá del HTF que se envía al bloque de
potencia al quedar este sistema fuera del modelo, por lo que es preciso
darle algún valor. * Caudal másico en entrada al campo solar: Nuestro
código calculará el caudal necesario para obtener la temperatura deseada
a la salida del lazo pero también empleará el caudal calculado por SAM
para comparar ambos. Por otro lazo, entre las diferencias que existen
entre nuestro código y SAM está que éste último considera nulo el caudal
en el campo solar durante las horas nocturnas. En las plantas reales
esto no es así, pues es necesario mantener cierto caudal en los lazos
para evitar que el HTF descienda por debajo de su punto de congelación.
El caudal mínimo empleado en cada lazo es de 1,7 kg/s, lo cual sumo 204
kg/s para un total de 120 lazos. Por este motivo, en nuestra simulación
se emplea este caudal mínimo durante las horas nocturnas y se calculan
las pérdidas en base a él. Si en algún momento SAM ha calculado un
cadual inferior a 204 kg/s sustituiremos ese valor por nuestro caudal
mínimo. * El resto de datos es idéntico para ambas plantas. SAM no
emplea las presiones del HTF en ningún momento de su cálculo. En nuestro
código se ha dejado esta posibilidad abierta de cara a incorporar, en un
futuro, un cálculo de pérdidas de carga en tuberías. Por este motivo
asignamos a todo el campo solar una presión de 20 bar, suficiente para
mantener el HTF en estado de líquido saturado durante toda la
simulación.

Tras ejecutar nuestra simulación estos son los datos más relevantes:

SAM

{Simulation}

{Benchmark}

Potencia térmica absorbida por el campo solar {[}GWh/año{]}

470,0

466,0

0,9\%

466,0

0,9\%

Potencia térmica a la salida del campo solar {[}GWh/año{]}

404,6

420,2

-3,9\%

417,2

-3,1\%

Potencia perdida en los HCEs de campo solar {[}GWh/año{]}

65,4

75,7

-15,7\%

77,6

-18,7\%

En comparación con SAM, nuestro programa

    \hypertarget{resultados-de-la-simulaciuxf3n.}{%
\subsection{Resultados de la
simulación.}\label{resultados-de-la-simulaciuxf3n.}}

\hypertarget{resultados-para-diferentes-configuraciones}{%
\subsubsection{Resultados para diferentes
configuraciones}\label{resultados-para-diferentes-configuraciones}}

\begin{itemize}
\tightlist
\item
  COMPARACIÓN ENTRE DATOS REALES, SAM Y PROGRAMA
\item
  TEMPERATURAS Y RENDIMIENTO EN UN MISMO LAZO
\item
  SIMULACIÓN CON OTRO HCE, SCA, FLUIDO
\end{itemize}

\hypertarget{conclusiones-y-trabajo-futuro}{%
\subsection{Conclusiones y trabajo
futuro}\label{conclusiones-y-trabajo-futuro}}

    

    \begin{tcolorbox}[breakable, size=fbox, boxrule=1pt, pad at break*=1mm,colback=cellbackground, colframe=cellborder]
\prompt{In}{incolor}{ }{\boxspacing}
\begin{Verbatim}[commandchars=\\\{\}]

\end{Verbatim}
\end{tcolorbox}


    % Add a bibliography block to the postdoc
    
    
    
\end{document}
